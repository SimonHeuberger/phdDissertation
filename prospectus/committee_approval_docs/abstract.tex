\documentclass[11pt]{article}
\usepackage[top=1in, bottom=1in, right=1in, left=1in]{geometry} % margins
\usepackage[dvipsnames]{xcolor}
\usepackage{setspace}
\usepackage{booktabs}
\usepackage[toc]{appendix} % appendix part of table of contents
\usepackage{titlesec}
\usepackage{float} % position tables exactly where I want them
\usepackage{amsmath}
\usepackage{graphicx}
\usepackage{dcolumn}
\usepackage{changepage}
\usepackage{fixltx2e}
\usepackage{setspace}
\usepackage{url} % web links
\usepackage{listings}
\usepackage{rotating}
\usepackage[final]{changes} % track changes
\usepackage{bbding}
\usepackage{paralist} % for \compactitem
\usepackage{array}
\usepackage{titling} % to remove space between title and date in \maketitle
\usepackage{soul} % to highlight text with \hl (also wraps the text) -- if including a link, wrap link in \hbox{}
\usepackage{hyperref} % for links
\hypersetup{
    colorlinks = false,
    allcolors = blue
} % activates blue color for all links
\colorlet{Changes@Color}{red}
\errorcontextlines 10000
\newcolumntype{.}{D{.}{.}{-1}}
\newcolumntype{d}[1]{D{.}{.}{#1}}
\lstset{
basicstyle=\small\ttfamily,
columns=flexible,
breaklines=true
}
\newcommand{\bibtex}{\textsc{Bib\TeX}}
\setremarkmarkup{(#2)}

\renewcommand\appendixname{Appendix}
\renewcommand\appendixpagename{Appendix}
\renewcommand\appendixtocname{Appendix}
\renewcommand\maketitlehookc{\vspace{-8ex}} % moves date up in \maketitle
\newlength{\mylen} % next 4 lines make that stupid bullet sign smaller and raise it to the same vertical height as the normal \bullet
\setbox1=\hbox{$\bullet$}\setbox2=\hbox{\tiny$\bullet$}
\setlength{\mylen}{\dimexpr0.5\ht1-0.5\ht2}
\renewcommand\labelitemi{\raisebox{\mylen}{\tiny$\bullet$}} 
\setlength{\droptitle}{-5em} % moves title up in \maketitle
\setlength\parindent{1cm}
\setremarkmarkup{(#2)}
\titleformat*{\section}{\normalsize\bfseries} % makes section titles a bit smaller but still bold
\titleformat*{\subsection}{\normalsize\itshape} % makes subsection titles a bit smaller but still bold
\titlespacing*{\section}{0pt}{0.4cm}{0cm} % reduces space before and deletes space after section title
\titlespacing*{\subsection}{0pt}{0.4cm}{0cm} % reduces space before and deletes space after subsection title
\graphicspath{ {figures/} } % loads figures from the "figures" folder (for aesthetics)
\bibliographystyle{chicago}
\newenvironment{coi}{\compactitem}{\endcompactitem} % renaming so I can type \begin{coi} instead of \begin{compactitem}
\usepackage{helvet} % next two lines use Arial font
\renewcommand{\familydefault}{\sfdefault}

% timeline:
\usepackage[utf8]{inputenc}
\usepackage[TS1,T1]{fontenc}
\usepackage{fourier, heuristica}
\usepackage{array, booktabs}
\usepackage{graphicx}
\usepackage{colortbl}
\usepackage{caption}
\DeclareCaptionFont{blue}{\color{NavyBlue}}
\newcommand{\foo}{\color{NavyBlue}\makebox[0pt]{\textbullet}\hskip-0.5pt\vrule width 1pt\hspace{\labelsep}}

\usepackage{natbib}
\bibliographystyle{chicago}



\title{Abstract of Dissertation Proposal}

\date{}

\begin{document}

\maketitle

\vspace{-1.6cm}


\section*{Scope and Significance of Research Topic}

\vspace{0.3cm}
I study American Politics and Quantitative Methods. I specialize in surveys. My dissertation will make a vital contribution to survey research in both fields. It has two intertwined objectives:

\vspace{-0.2cm}

\begin{enumerate}[I.]
	\item Improve balance in online survey experiments
	 \vspace{-0.3cm}
	\item Uncover the source of framing effects in public opinion surveys
\end{enumerate}

\vspace{-0.2cm}

First, I will build a software tool using the statistical software R. This tool applies a sophisticated randomization method to online survey experiments that greatly improves the method provided by the popular online survey design platform Qualtrics. An ever-increasing number of survey experiments are conducted online. At the moment, however, it is very difficult for researchers to field online survey experiments with sophisticated randomization if they do not have a substantial amount of money to purchase large samples. This is hugely problematic as it prevents particularly junior researchers with little money from conducting methodologically sound experiments and advancing knowledge. Currently, online survey experiments with small to medium sample sizes suffer from serious randomization bias, which can render their results meaningless. My software tool fills this gap. It enables researchers with little money to field small-sample online survey experiments and still get methodologically sound, meaningful results, which they otherwise would not be able to achieve. This is an important contribution to political science experimentation and to survey experimentation overall.

Second, I will design a questionnaire for an online survey experiment that investigates the source of framing effects in public opinion. Research has shown that some frames influence how people see issues, but we do not know the underlying source behind these framing effects, i.e. why they are influential. Why is it possible to shift people's opinions with some frames? I attempt to answer this question by applying theoretical moralization claims to empirical framing research. I hypothesize that moralization, i.e. presenting an issue in moral frames, is at the heart of these framing effects. To test this hypothesis, I will field the designed survey twice: Once with Qualtrics' basic randomization, and once using the sophisticated randomization tool I develop. I will then examine the balances in both surveys, thereby demonstrating the quality of my tool, and analyze the substantive results on the importance of moral frames. Substantively, my dissertation sheds much needed light on a very important question: Are moral arguments central to the formation of public opinion?. My experiment will give an answer to this question and thus make an important contribution to research on a fundamental component of human behavior.






\section*{Brief Summary of Current Literature of Research Topic}

\vspace{0.3cm}

The first section describes the statistical background of experimental randomization and explains the basics of the sophisticated randomization method (``sequential blocking") I employ. My software tool will provide an interface to apply this method to online survey experiments. The second section briefly describes the current literature on framing experiments in American Politics.

\vspace{0.3cm}

In order to gather valid causal inference from experiments, researchers need to randomize assignment of participants to treatment and control \citep{king_a-politically_2007,imai_2009_essential}. In complete randomization, each participant is assigned to treatment or control based on an independently generated random number \citep{lachin_1988_properties,king_designing_1994} -- it is the equivalent of flipping a coin each time. Complete randomization promotes balance as the number of participants increases -- the larger the sample, the better complete randomization works \citep{moore_2012_multivariate,fox_applied_2015}. In online survey experiments, however, sample sizes are usually small \citep{imai_2008_misunderstandings}. The technique of blocking helps ensure balance in small samples \citep{moore_blocking_2013}. Blocking occurs when a researcher uses observed covariates -- age, party ID, income etc. -- to create pre-assignment groups of similar participants. This dramatically improves balance in small-sample experiments \citep{epstein_2002_rules}. Sequential blocking is a special form of blocking. In simple terms, there are are two different starting positions for the researcher who wants to conduct an experiment: (1) The researcher knows the characteristics of all participants at the time of randomization, and (2) the researcher does not know the characteristics of all participants at the time of randomization. In position (1), `normal' blocking is possible. In position (2), the researcher has to use sequential blocking.  The researcher only has information on the units that have already arrived, but not future units. Since participants thus `trickle in' to be assigned and are not all present at the time of randomization, different blocking methods are required to exploit the background information that is available \citep{urdan_statistics_2010,boas_fielding_2013,imai_quantitative_2018}.

\vspace{0.3cm}

Barack Obama presented an outline of the Affordable Care Act in the summer of 2009. The content of the reform was put online for everyone to see, but since the administration was still working on details, it refrained from actively communicating it. Press releases simply stated that the ACA would expand coverage and lower health care costs. This hesitancy turned out to be a big mistake. In July, support for the ACA hovered around 43 percent. Then Sarah Palin, John McCain's choice for running mate in 2008, posted the following statement on Facebook on August 7\textsuperscript{th}: ``The America I know and love is not one in which my parents or my baby with Down Syndrome will have to stand in front of Obama's `death panel' so his bureaucrats can decide, based on a subjective judgment of their `level of productivity in society'\,'' \citep{palin_statement_2009}. Palin implied that federal government workers would be able to refuse treatment to any patients and thus `decide their fate'. Over the next two weeks, support for the ACA dropped to 35 percent while opposition rose to 52 percent. Republican lawmakers jumped at the opportunity and repeated the claim of `death panels' whenever possible. The reform never recovered from this drop. In December 2009, four months after the statement, support and opposition were virtually identical to August. While the public was still uncertain about the exact contents of the law, Palin had asserted that it would include a Big Brother type panel that decided whether people would live or die. This drowned out any efforts by the Obama administration to show the law as a cost-reducing reform. Palin's frame of the ACA, in other words, drastically influenced public opinion of the reform.

Framing is the practice of presenting an issue to affect the way people see it \citep{aaroe_investigating_2011, druckman_evaluating_2001,tversky_framing_1981}. We learn about healthcare reform through articles, reports, speeches, commercials and social media. This mediated communication possesses tremendous potential influence on our perception of political issues \citep{iyengar_framing_1996, gross_framing_2008,kam_risk_2010}. Framing research has established that a variety of frames substantively influence how people view and think about issues, such as the ACA \citep{ price_switching_1997,andsager_how_2000, callaghan_introduction:_2005, entman_framing:_1993, entman_projections_2004, gamson_media_1989, vreese_effects_2004, pan_framing_1993, slothuus_political_2010, sniderman_structure_2004}. But we do not know why these frames elicit these effects. A major challenge for framing research thus ``concerns the identification of factors that make a frame strong'' \citep[p.~116]{chong_framing_2007}. My dissertation will fill this gap, thereby making an important contribution to understanding how people form and change their opinion about political issues. 





\section*{Proposed Research Question and Hypothesis}

\vspace{0.3cm}

As stated above, I will develop a software that enables researchers to field small-sample online survey experiments with statistically valid results. This is currently not possible. The only available tool to implement in online survey experiments is complete randomization. Providers such as the popular online survey design platform Qualtrics offer complete randomization as part of their interface. As mentioned above, complete randomization is not a problem for large samples. Large samples, however, cost a lot of money. At the moment, it is thus very difficult for researchers to field online survey experiments with sophisticated randomization if they do not have a substantial amount of money to purchase large samples. This is hugely problematic as it prevents particularly junior researchers with little money from conducting methodologically sound experiments and advancing knowledge. The academic community and political science knowledge overall is worse off because junior researchers frequently are prevented from conducting online survey experiments, solely due to financial restrictions. Currently, online survey experiments with small to medium sample sizes suffer from serious randomization bias, which can render their results meaningless. My software tool fills this gap. My objective is to enable researchers with little money to field small-sample online survey experiments and still get methodologically sound, meaningful results, which they otherwise would not be able to achieve. This is an important contribution to political science experimentation and to survey experimentation overall.

\vspace{0.3cm}

As stated above, the substantive part of my dissertation will address the sources of frame strength in political framing.  We know from experimental research that some frames influence people's issue positioning, but we do not know why this is possible. I attempt to answer this question by applying theoretical moralization claims to empirical framing research. Moralization literature asserts that moral arguments are essential to how people make sense of the world  \citep{mooney_public_2001}. Moral arguments are seen as fair, near-universal standards of truth. This stands in contrast to pragmatic arguments, which are considered less influential and of a financially pragmatic nature. Scholars argue that moral arguments trump pragmatic arguments in persuasive power because they achieve a higher emotional connection \citep{haidt_moral_2003}. Following this argument, I hypothesize that moralization, i.e. making moral arguments, lies at the heart of frame strength. I conduct an online survey experiment to test this hypothesis. This sheds much needed light on a very important question: Are moral arguments as powerful as the literature suggests? Scholars tend to convene that moral conviction represents an important force that guides the development of public opinion \citep{skitka_moral_2005,skitka_moral_2011}, but this has not been extensively tested. If my experiment reveals stronger moral frames, it provides statistically sound empirical proof for the power of moral arguments. If my experiment reveals stronger pragmatic frames, it raises doubts over the claimed importance of moral conviction. My experiment thus makes an important contribution to research on a fundamental component of human behavior.




\section*{Proposed Research Methodology}

\vspace{0.3cm}

Creating the sequential blocking software will involve lengthy computational code development and subsequent statistical testing to ensure its functionality. Once that is completed, I will apply the software tool in several thousand computational statistical simulations with artificial data and compare the sequentially blocked balance to the results from several thousand computational statistical simulations of the same data with complete randomization. Similarly, I will apply the tool and complete randomization in external experimental survey data from published articles in peer-reviewed journals. Finally, I conduct my own original online survey experiment on moral framing (described in more detail below), once using the tool and once using complete randomization. Subsequently, I conduct statistical tests to demonstrate the improvements provided by the tool. All methods to develop, test and assess the tool are thus quantitative in nature.

\vspace{0.3cm}

To obtain an objective summary of the framing effects found in the literature, I will conduct a statistical meta analysis of all experimental framing research conducted up to the present day. A meta analysis corrects any bias and/or errors present in individual studies. To acquire an accurate representation of what people consider to be moral and pragmatic arguments, I will conduct a focus group with a commercial research company in DC. Focus groups are designed to obtain qualitative information that cannot be collected through quantitative means. I will use the results from the quantitative meta analysis and the qualitative focus group to design a survey questionnaire that assesses the effects of moral and pragmatic frames on several political issues, among them minimum wage, healthcare, and housing. The questionnaire will also collect standard demographic information. I will pre-test the designed questionnaire to provide experimental proof (in addition to the qualitative insights from the focus group) that the designed frames indeed correspond with people's views on what constitutes moral and pragmatic arguments. I will field the final questionnaire online twice: Once with Qualtrics' complete randomization, and once with sequential blocking provided by my tool. Finally, I will quantitatively analyze the substantive results of both surveys and interpret them in terms of their significance for the formation and changing of public opinion in political science. The analysis of moral arguments as a source of frame strength will thus involve a combination of qualitative and quantitative methods. 



\section*{Analytic Plan for Proposed Data Collection}


\begin{table}[H]
\renewcommand\arraystretch{1.4}\arrayrulecolor{NavyBlue}
\captionsetup{singlelinecheck=false, font=blue, labelfont=sc, labelsep=quad}
\caption*{Timeline June 2018 -- April 2020}\vskip -1.5ex
\begin{tabular}{@{\,}r <{\hskip 2pt} !{\foo} >{\raggedright\arraybackslash}p{13.3cm}}
\toprule
\addlinespace[1.5ex]
\textit{June 2018} & Develop sequential blocking software tool using R \\
\textit{September} & Conduct meta-analysis of experimental framing research \hspace{10cm} Conduct statistical tests on R tool\\
\textit{December} & Design protocol for framing focus group \hspace{10cm} Apply R tool in simulations\\
\textit{February 2019} & Conduct focus group  \hspace{10cm} Apply R tool in external experimental survey data\\
\textit{March} & Analyze results of focus group\\
\textit{April} & Implement focus group results \hspace{10cm} Design framing survey questionnaire\\
\textit{June} & Pre-test questionnaire online\\
\textit{July} & Implement insights from pre-test \hspace{10cm} Field survey online with complete randomization \hspace{10cm} Field survey online with sequential blocking \\
\textit{August} & Analyze substantive survey results on the power of moral frames\\
\textit{October} & Conduct statistical balance tests to demonstrate R tool quality\\
\textit{November} & Write up all results\\
\textit{April 2020} & Defend dissertation\\
\end{tabular}
\end{table}








\clearpage

\bibliography{/Users/simonheuberger/Dropbox/work/bib_files/all_together.bib}




\end{document}

