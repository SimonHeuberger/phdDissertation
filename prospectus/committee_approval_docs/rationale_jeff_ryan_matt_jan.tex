\documentclass[11pt]{article}
\usepackage[top=1in, bottom=1in, right=1in, left=1in]{geometry} % margins
\usepackage[dvipsnames]{xcolor}
\usepackage{setspace}
\usepackage{booktabs}
\usepackage[toc]{appendix} % appendix part of table of contents
\usepackage{titlesec}
\usepackage{float} % position tables exactly where I want them
\usepackage{amsmath}
\usepackage{graphicx}
\usepackage{dcolumn}
\usepackage{changepage}
\usepackage{fixltx2e}
\usepackage{setspace}
\usepackage{url} % web links
\usepackage{listings}
\usepackage{rotating}
\usepackage[final]{changes} % track changes
\usepackage{bbding}
\usepackage{paralist} % for \compactitem
\usepackage{array}
\usepackage{titling} % to remove space between title and date in \maketitle
\usepackage{soul} % to highlight text with \hl (also wraps the text) -- if including a link, wrap link in \hbox{}
\usepackage{hyperref} % for links
\hypersetup{
    colorlinks = false,
    allcolors = blue
} % activates blue color for all links
\colorlet{Changes@Color}{red}
\errorcontextlines 10000
\newcolumntype{.}{D{.}{.}{-1}}
\newcolumntype{d}[1]{D{.}{.}{#1}}
\lstset{
basicstyle=\small\ttfamily,
columns=flexible,
breaklines=true
}
\newcommand{\bibtex}{\textsc{Bib\TeX}}
\setremarkmarkup{(#2)}

\renewcommand\appendixname{Appendix}
\renewcommand\appendixpagename{Appendix}
\renewcommand\appendixtocname{Appendix}
\renewcommand\maketitlehookc{\vspace{-8ex}} % moves date up in \maketitle
\newlength{\mylen} % next 4 lines make that stupid bullet sign smaller and raise it to the same vertical height as the normal \bullet
\setbox1=\hbox{$\bullet$}\setbox2=\hbox{\tiny$\bullet$}
\setlength{\mylen}{\dimexpr0.5\ht1-0.5\ht2}
\renewcommand\labelitemi{\raisebox{\mylen}{\tiny$\bullet$}} 
\setlength{\droptitle}{-5em} % moves title up in \maketitle
\setlength\parindent{1cm}
\setremarkmarkup{(#2)}
\titleformat*{\section}{\normalsize\bfseries} % makes section titles a bit smaller but still bold
\titleformat*{\subsection}{\normalsize\itshape} % makes subsection titles a bit smaller but still bold
\titlespacing*{\section}{0pt}{0.4cm}{0cm} % reduces space before and deletes space after section title
\titlespacing*{\subsection}{0pt}{0.4cm}{0cm} % reduces space before and deletes space after subsection title
\graphicspath{ {figures/} } % loads figures from the "figures" folder (for aesthetics)
\bibliographystyle{chicago}
\newenvironment{coi}{\compactitem}{\endcompactitem} % renaming so I can type \begin{coi} instead of \begin{compactitem}
\usepackage{helvet} % next two lines use Arial font
\renewcommand{\familydefault}{\sfdefault}

% timeline:
\usepackage[utf8]{inputenc}
\usepackage[TS1,T1]{fontenc}
\usepackage{fourier, heuristica}
\usepackage{array, booktabs}
\usepackage{graphicx}
\usepackage{colortbl}
\usepackage{caption}
\DeclareCaptionFont{blue}{\color{NavyBlue}}
\newcommand{\foo}{\color{NavyBlue}\makebox[0pt]{\textbullet}\hskip-0.5pt\vrule width 1pt\hspace{\labelsep}}

\usepackage{natbib}
\bibliographystyle{chicago}




\title{Rationale for Inclusion of Proposed Dissertation Committee Members}

\date{}

\begin{document}

\maketitle

\vspace{-1.6cm}

\section*{Jeff Gill (Chair)}

\vspace{0.3cm}

Professor Gill researches and teaches statistical methodology and software in political science, statistics, and mathematics. I have been working with Professor Gill as his research assistant since he joined American University in the summer of 2017. In this capacity, I replicate data submissions to the journal \textit{Political Analysis}, of which Professor Gill is the editor-in-chief. We are also in the process of publishing an article and a book chapter on Bayesian models. As the chair of my dissertation committee, Professor Gill will supervise the overall completion process and my weekly progress. I will continue to work for Professor Gill as his research assistant in the coming years, so we will interact on a daily basis. Substantively, Professor Gill's role will focus on the development of the statistical software tool, the quantitative applications of the tool, as well as the usage of all quantitative methods overall. Together with Professor Moore (below), Professor Gill will be primarily engaged in all quantitative aspects of my dissertation.

\section*{Ryan T. Moore}

\vspace{0.3cm}

Professor Moore researches and teaches statistical political methodology and software, with applications in social and political behavior. I have been working with Professor Moore for the past three years as a student in his classes on statistical political methodology, as his research assistant in 2016/17 where we taught GOVT-310 Introduction to Political Research, and on collaborative papers and classes afterwards, such as an R workshop in the fall of 2017. We are currently writing a paper on Congressional candidate characteristics using conjoint survey methodology. Professor Moore specializes in the development and implementation of methods for political experiments and causal inference. Substantively, Professor Moore's role in the dissertation process will focus on the development of the statistical software tool, the quantitative applications of the tool, as well as the experimental implementation in the online survey. He will provide counsel on the usage of all quantitative methods together with Professor Gill.

\section*{Matthew Wright}

\vspace{0.3cm}

Professor Wright researches and teaches American politics, public opinion, political psychology, political behavior and political methodology. I have been working with Professor Wright for the past three years as a student in his classes on American politics and research design, as his research assistant in 2015/16, and afterwards on collaborative projects. We also presented a paper on morality in political arguments at the APSA (American Political Science Association) Conference together in 2016. Professor Wright is an expert in political psychology and public opinion. Substantively, Professor Wright's role in the dissertation process will focus on the structure of the framing meta-analysis, the design and execution of the focus group, as well as the design and testing of the survey questionnaire and its subsequent substantive analysis. Together with Professor Leighley (below), Professor Wright will be primarily engaged in all qualitative and substantive aspects of my dissertation.

\section*{Jan E. Leighley}

\vspace{0.3cm}

Professor Leighley researches and teaches American politics, political behavior, public opinion and voter turnout. I have been working with Professor Leighley since the summer of 2016, where I assisted her in the teaching of GOVT-310 Introduction to Political Research. We have subsequently presented a paper on the effects of emotion on voter turnout at the SPSA (Southern Political Science Association) Conference together in 2017. Professor Leighley is an authority on the impact of communication on public opinion. Substantively, Professor Leighley's role in the dissertation process will focus on the structure of the framing meta-analysis, the design and execution of the focus group, as well as the design and testing of the survey questionnaire and its subsequent substantive analysis. She will do so in collaboration with Professor Wright.





\end{document}

