\documentclass[11pt]{article}
\usepackage[top=1in, bottom=1in, right=1in, left=1in]{geometry} % margins
\usepackage[dvipsnames]{xcolor}
\usepackage{setspace}
\usepackage{booktabs}
\usepackage[toc]{appendix} % appendix part of table of contents
\usepackage{titlesec}
\usepackage{float} % position tables exactly where I want them
\usepackage{amsmath}
\usepackage{graphicx}
\usepackage{dcolumn}
\usepackage{changepage}
\usepackage{fixltx2e}
\usepackage{setspace}
\usepackage{url} % web links
\usepackage{listings}
\usepackage{rotating}
\usepackage[final]{changes} % track changes
\usepackage{bbding}
\usepackage{paralist} % for \compactitem
\usepackage{array}
\usepackage{titling} % to remove space between title and date in \maketitle
\usepackage{soul} % to highlight text with \hl (also wraps the text) -- if including a link, wrap link in \hbox{}
\usepackage{hyperref} % for links
\hypersetup{
    colorlinks = false,
    allcolors = blue
} % activates blue color for all links
\colorlet{Changes@Color}{red}
\errorcontextlines 10000
\newcolumntype{.}{D{.}{.}{-1}}
\newcolumntype{d}[1]{D{.}{.}{#1}}
\lstset{
basicstyle=\small\ttfamily,
columns=flexible,
breaklines=true
}
\newcommand{\bibtex}{\textsc{Bib\TeX}}
\setremarkmarkup{(#2)}

\renewcommand\appendixname{Appendix}
\renewcommand\appendixpagename{Appendix}
\renewcommand\appendixtocname{Appendix}
\renewcommand\maketitlehookc{\vspace{-8ex}} % moves date up in \maketitle
\newlength{\mylen} % next 4 lines make that stupid bullet sign smaller and raise it to the same vertical height as the normal \bullet
\setbox1=\hbox{$\bullet$}\setbox2=\hbox{\tiny$\bullet$}
\setlength{\mylen}{\dimexpr0.5\ht1-0.5\ht2}
\renewcommand\labelitemi{\raisebox{\mylen}{\tiny$\bullet$}} 
\setlength{\droptitle}{-5em} % moves title up in \maketitle
\setlength\parindent{1cm}
\setremarkmarkup{(#2)}
\titleformat*{\section}{\normalsize\bfseries} % makes section titles a bit smaller but still bold
\titleformat*{\subsection}{\normalsize\itshape} % makes subsection titles a bit smaller but still bold
\titlespacing*{\section}{0pt}{0.4cm}{0cm} % reduces space before and deletes space after section title
\titlespacing*{\subsection}{0pt}{0.4cm}{0cm} % reduces space before and deletes space after subsection title
\graphicspath{ {figures/} } % loads figures from the "figures" folder (for aesthetics)
\bibliographystyle{chicago}
\newenvironment{coi}{\compactitem}{\endcompactitem} % renaming so I can type \begin{coi} instead of \begin{compactitem}
\pagenumbering{gobble} % to suppress page numbers
\usepackage{helvet} % next two lines use Arial font
\renewcommand{\familydefault}{\sfdefault}

% timeline:
\usepackage[utf8]{inputenc}
\usepackage[TS1,T1]{fontenc}
\usepackage{fourier, heuristica}
\usepackage{array, booktabs}
\usepackage{graphicx}
\usepackage{colortbl}
\usepackage{caption}
\DeclareCaptionFont{blue}{\color{NavyBlue}}
\newcommand{\foo}{\color{NavyBlue}\makebox[0pt]{\textbullet}\hskip-0.5pt\vrule width 1pt\hspace{\labelsep}}

\usepackage{natbib}
\bibliographystyle{chicago}
 
\title{\textbf{\large{\textcolor{NavyBlue}{PROJECT DESCRIPTION}\\DOCTORAL STUDENT RESEARCH\\\vspace{-0.2cm}SCHOLARSHIP}}}

\date{}

\begin{document}

\maketitle

\vspace{-1.6cm}

\singlespacing

\section{Overall Objectives of the Dissertation}

\vspace{0.3cm}


% Expand first section with theory of what I think will happen in moralization, of what's going on there


I am a student of American Politics and Quantitative Methods. My dissertation will make a vital contribution in both fields. It thus has two overall objectives:

\vspace{-0.2cm}

\begin{enumerate}[I.]
	\item Improve treatment randomization in online survey experiments
	 \vspace{-0.3cm}
	\item Uncover the source of framing effects in public opinion surveys
\end{enumerate}

\vspace{-0.2cm}

These objectives are intertwined. First, I will build a software tool using the statistical software R. This tool applies a sophisticated randomization method to online survey experiments that greatly improves the method provided by the popular online survey design platform Qualtrics. Second, I will design a questionnaire for an online survey experiment that investigates the source of framing effects in public opinion. Research has shown that some frames influence how people see issues, but we do not know why these frames are influential. I hypothesize that moralization, i.e. presenting an issue in moral frames, is at the heart of these framing effects. To test this hypothesis, I will field the designed survey twice: Once with Qualtrics' basic randomization, and once using the sophisticated randomization tool I developed. I will then examine the balances in both surveys, thereby demonstrating the quality of my tool, and analyze the substantive results on the importance of moral frames. Section 2 will provide further details.

\section{Project Design and Procedures}

\vspace{0.3cm}

The majority of online surveys use Qualtrics, a service to design questionnaires \citep{boas_fielding_2013}. Qualtrics also offers the option to randomize assignment to treatment groups through flip-a-coin randomization \citep{urdan_statistics_2010}. For each respondent, the computer flips a coin to decide which treatment group to assign her to. The goal of randomization is to make all treatment groups look the same on average. This is called balance. Using flip-a-coin for large samples results in balance based on the law of large numbers. Using it for small samples, however, can result in serious imbalance. It can easily be that the treatment groups will not look the same. This can render experimental results useless \citep{imai_quantitative_2018,king_designing_1994,fox_applied_2015} Large samples can be hard to get because they cost a lot of money. These financial constraints are exacerbated in survey experiments, where the overall sample size is split across several treatment groups. Researchers with limited funds can often only conduct small-sample online experiments. Because of Qualtrics' insufficient randomization, these experiments can produce substantively useless results. This is hugely problematic. I develop a software tool that fills this gap by employing a sophisticated method of randomization called sequential blocking. Sequential blocking assigns each respondent based on information from previously assigned respondents. Research has shown that this method greatly improves balance \citep{moore_blocking_2013}. There is currently no way to incorporate this method into online surveys. My tool changes that. By directly feeding sequential blocking into Qualtrics, it enables researchers to field statistically sound small-sample online survey experiments, which they would not be able to achieve otherwise.

Once I have developed this tool, I will design a questionnaire to examine the effect of moral arguments, particularly moral framing, on public opinion. Framing is the practice of presenting an issue to affect people's support for it \citep{druckman_evaluating_2001,gross_framing_2008}. Presenting Obama's health care reform as an invasion of privacy achieves less support for it than presenting it as help for millions of uninsured Americans, for instance. Research has shown that frames influence how people view issues \citep{iyengar_framing_1996,tversky_framing_1981}. But what are the factors that make frames so influential? Research does not have an answer to this question \citep{ryan_reconsidering_2014}. I address this gap by investigating whether moralization, i.e. presenting an issue in moral terms, lies at the bottom of frame strength. Theoretical moralization literature claims that moral arguments (based on fairness) are more powerful than pragmatic arguments (based on financial aspects) \citep{tatalovich_moral_2011}. I apply this claim to framing and investigate whether moral frames elicit larger effects than pragmatic frames. I will design a survey questionnaire with five treatment groups on the issues of minimum wage, healthcare, housing, and eminent domain. The treatment groups will contain supporting and opposing moral and pragmatic frames. The questionnaire will also collect standard demographic information. I will pre-test the questionnaire to ensure that the frames correspond with people's views on what constitutes a moral and a pragmatic argument. Such a pre-test is required practice in political science survey experiments.

I will field the pre-tested questionnaire online twice: Once with Qualtrics' flip-a-coin randomization, and once with sequential blocking provided by my tool. I will use Amazon's online platform MTurk to recruit respondents for the pre-test and for the surveys. MTurk is a service where researchers can host tasks to be completed by anonymous respondents. Respondents receive financial compensation for their work and Amazon collects a commission. MTurk samples have been shown to be internally valid in survey experiments \citep{hauser_attentive_2016}. The use of MTurk in political science experiments has increased dramatically over the past decade and is now common practice \citep{berinsky_evaluating_2012}. After I have collected the responses, I will demonstrate the improvements produced by my method by comparing the balances achieved in the two otherwise identical surveys. I will also analyze the results of the sophistically randomized survey on the importance of moral frames. 



\section{The Role of this Project to Completing the Dissertation}

\vspace{0.3cm}

Conducting these two surveys is crucial to completing the dissertation. Without original experimental survey data, I am unable to demonstrate the quality of my randomization method. Providing empirical proof with original data is key to publishing statistical procedures in academia. My contribution to statistical survey methods would be stuck in the development phase and I would not be able to present my method to the academic community. Without original experimental survey data, I am also unable to empirically analyze the importance of moral arguments in political discourse. My substantive contribution to political science would be reduced to mere theoretical musings about the influence of moral frames, without any form of causal inference. 


\section{Project Timeline}


\begin{table}[H]
\renewcommand\arraystretch{1.4}\arrayrulecolor{NavyBlue}
\captionsetup{singlelinecheck=false, font=blue, labelfont=sc, labelsep=quad}
\caption*{Timeline for Grant Period May 2018 -- April 2019}\vskip -1.5ex
\begin{tabular}{@{\,}r <{\hskip 2pt} !{\foo} >{\raggedright\arraybackslash}p{13.3cm}}
\toprule
\addlinespace[1.5ex]
\textit{June} & Computational development of sophisticated randomization tool using R \\
\textit{October} & Design survey questionnaire and pre-test on MTurk \\
\textit{January} & Field survey with Qualtrics flip-a-coin randomization on MTurk \hspace{10cm} Field survey with sequential blocking on MTurk \\
\textit{February} & Demonstrate balance improvements of sequential blocking  \hspace{2.5cm} Analyze substantive survey results on the power of moral frames \\
\end{tabular}
\end{table}

\section{The Importance of these Funds to the Completion of the Dissertation}

\vspace{0.3cm}

There are no costs involved in the development of the software. I will compute the sophisticated randomization method entirely with open source resources. I am applying for this scholarship to raise the funds to cover the costs of pre-testing the survey questionnaire and of fielding the survey online twice. These costs amount to \$4,830 and are detailed in the Budget Justification.

Without the pre-test, we would not know whether some frames are potentially flawed and could not address this before fielding the actual survey. Without the necessary funds, I could not conduct this pre-test, which in turn could invalidate the quality of the questionnaire, which in turn would jeopardize any causal inference I intend to make in this dissertation. Furthermore, without the funds to subsequently field the survey, there simply would not be a dissertation. Both the improvement of online survey experimental methodology and the substantive investigation of the power of moral arguments in political discourse depend crucially on the fielding of this survey. Without the funds to do so, I could not complete my dissertation. I am also seeking external funding to cover these costs, for instance through the NSF Dissertation Improvement Grant.

\section{The Significance or Expected Impact of the Dissertation}

\vspace{0.3cm}

An ever-increasing number of survey experiments are conducted online \citep{mutz_population-based_2011}. At the moment, however, it is very difficult for researchers to field online survey experiments with sophisticated randomization if they do not have a substantial amount of money to purchase large samples. This is hugely problematic as it prevents particularly junior researchers with little money from conducting methodologically sound experiments and advancing knowledge. Currently, online survey experiments with small to medium sample sizes suffer from serious randomization bias, which can render their results meaningless. My method fills this gap. It enables researchers with little money to field small-sample online survey experiments and still get methodologically sound, meaningful results, which they otherwise would not be able to achieve. This is an important contribution to political science experimentation and to survey experimentation overall.

We know that some frames influence people's issue positioning. However, we do not know why this is the case. We do not know the underlying source behind these framing effects. Why is it possible to shift people's opinions with frames? I attempt to answer this question by applying theoretical moralization claims to empirical framing research. Moralization literature asserts that moral arguments are essential to how people make sense of the world around them \citep{mooney_public_2001}. Moral arguments are seen as fair, near-universal standards of truth. This stands in contrast to pragmatic arguments, which are considered much less influential and of a financially pragmatic nature. Scholars further argue that moral arguments trump pragmatic arguments in persuasive power because they achieve a higher emotional connection \citep{haidt_moral_2003}. Following this argument, I hypothesize that moralization, i.e. giving moral arguments, lies at the heart of frame strength. My survey experiment tests this hypothesis. It thus sheds much needed light on a very important question: Are moral arguments as powerful as the literature suggests? Scholars tend to convene that moral conviction represents an important force that guides the development of public opinion \citep{skitka_moral_2005,skitka_moral_2011}, but this has not been extensively tested. If my experiment reveals stronger moral frames, it provides statistically sound empirical proof for the power of moral arguments. If my experiment reveals stronger pragmatic frames, it raises doubts over the claimed importance of moral conviction. My experiment thus makes an important contribution to research on a central component of human behavior.

\clearpage

\bibliography{/Users/simonheuberger/Dropbox/work/bib_files/all_together.bib}
 
\end{document}