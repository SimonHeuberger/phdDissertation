\documentclass[11pt]{article}
\usepackage[top=1in, bottom=1in, right=1in, left=1in]{geometry} % margins
\usepackage[dvipsnames]{xcolor}
\usepackage{setspace}
\usepackage{booktabs}
\usepackage[toc]{appendix} % appendix part of table of contents
\usepackage{titlesec}
\usepackage{float} % position tables exactly where I want them
\usepackage{amsmath}
\usepackage{graphicx}
\usepackage{dcolumn}
\usepackage{changepage}
\usepackage{setspace}
\usepackage{url} % web links
\usepackage{listings}
\usepackage{rotating}
\usepackage{natbib}
\usepackage[final]{changes} % track changes
\usepackage{bbding}
\usepackage{paralist} % for \compactitem
\usepackage{array}
\usepackage{titling} % to remove space between title and date in \maketitle
\usepackage{soul} % to highlight text with \hl (also wraps the text) -- if including a link, wrap link in \hbox{}
\errorcontextlines 10000
\newcolumntype{.}{D{.}{.}{-1}}
\newcolumntype{d}[1]{D{.}{.}{#1}}
\lstset{
basicstyle=\small\ttfamily,
columns=flexible,
breaklines=true
}
\newcommand{\bibtex}{\textsc{Bib\TeX}}
\setremarkmarkup{(#2)}

\renewcommand\appendixname{Appendix}
\renewcommand\appendixpagename{Appendix}
\renewcommand\appendixtocname{Appendix}
\renewcommand\maketitlehookc{\vspace{-8ex}} % moves date up in \maketitle
\newlength{\mylen} % next 4 lines make that stupid bullet sign smaller and raise it to the same vertical height as the normal \bullet
\setbox1=\hbox{$\bullet$}\setbox2=\hbox{\tiny$\bullet$}
\setlength{\mylen}{\dimexpr0.5\ht1-0.5\ht2}
\renewcommand\labelitemi{\raisebox{\mylen}{\tiny$\bullet$}} 
\setlength{\droptitle}{-5em} % moves title up in \maketitle
\setlength\parindent{1cm}
\setremarkmarkup{(#2)}
\graphicspath{ {figures/} } % loads figures from the "figures" folder (for aesthetics)
\newenvironment{coi}{\compactitem}{\endcompactitem} % renaming so I can type \begin{coi} instead of \begin{compactitem}

% timeline:
\usepackage[utf8]{inputenc}
\usepackage[TS1,T1]{fontenc}
\usepackage{fourier, heuristica}
\usepackage{array, booktabs}
\usepackage{graphicx}
\usepackage{colortbl}
\usepackage{caption}
\DeclareCaptionFont{blue}{\color{NavyBlue}}
\newcommand{\foo}{\color{NavyBlue}\makebox[0pt]{\textbullet}\hskip-0.5pt\vrule width 1pt\hspace{\labelsep}}
 
\usepackage{graphicx}
\usepackage{amsmath}
\usepackage{enumitem}
\setlength\parindent{24pt}
\usepackage{hyperref}
\usepackage{url}
\hypersetup{
  colorlinks = true,
  linkcolor=magenta, % color of internal links
  citecolor=magenta, % color of links to bibliography
  urlcolor=magenta, % color of external links
  bookmarksopen=true,
  pdfdisplaydoctitle=true
}
\usepackage{dcolumn}
\pagestyle{myheadings}

\titleformat{\section}{\normalfont\large\bfseries}{Chapter \thesection:}{1em}{}

% to have the page numbers at the bottom in the center
\usepackage{fancyhdr} 
\fancyhf{}
\renewcommand{\headrulewidth}{0pt}
\cfoot{\thepage}
\pagestyle{fancy}
 
\begin{document}

\begin{center}
\Large
\textcolor{NavyBlue}{What the People Think: Advances in Public Opinion Measurement Using Ordinal Variables}\\
\vspace{0.3cm}
\large
\sc{Abstract}
\end{center}

\doublespacing

Surveys are a central part of political science. Without surveys, we would not know what people think about political issues. Survey experiments further enable us to test how people react to given treatments. Surveys and survey experiments are only as good as the measurements and analytical techniques we as researchers employ, though. For one particular survey variable called ordinal variables, some of our current measurements and techniques are insufficient. Ordinal variables consist of ordered categories where the spacing between each category is uneven and not known. Ordinal variables are highly important because the most important predictor of political behavior is an ordinal variable: education. Ordinal example categories for education could be ``Some High School'', ``High School Graduate'', and ``Bachelor's Degree''. The literature currently often does not take the special nature of ordinal variables, i.e. their uneven spacing, into account. This could misrepresent the data and potentially distort survey results. It is important that we measure and use education and other ordinal variables correctly. My dissertation develops two methods to do so and applies them in original survey research. Chapter 1 develops a new method to improve the use of ordinal variables in the assignment of treatment in survey experiments. Chapter 2 develops a new method to treat missing survey data with ordinal variables. Chapter 3 applies both methods in an online survey experiment on political framing. 




\clearpage

\begin{center}
\Large
\textcolor{NavyBlue}{What the People Think: Advances in Public Opinion Measurement Using Ordinal Variables}\\
\vspace{0.3cm}
\large
\sc{Summary}
\end{center}


\section{Improving Blocking on Ordinal Variables in Survey Experiments}

\subsection{Background}

Survey experiments attempt to uncover treatment effects. To do so, all treatment groups need to look the same, i.e. they must be balanced. This can be achieved through random assignment. Complete randomization probabilistically results in balance based on the Law of Large Numbers. For small samples, however, it can result in imbalance \citep{holland_1986_statistics,rubin_1974_estimating}. Blocking, i.e. arranging participants in groups that are equal in terms of demographic information, guarantees balance \citep{moore_blocking_2013,imai_2009_essential}. However, researchers often simplify blocking on ordinal variables by making them numeric without justification \citep{urdan_statistics_2010,imai_quantitative_2018}. 

One of the most important predictors in political science is an ordinal variable: education. It is widely established that education represents one of the major driving forces behind political behavior in the U.S., such as turnout or donations \citep{dawood_campaign_2015,leighley_who_2014,druckman_how_2013}. Ordinal variables consist of ordered categories where the spacing between each category is uneven and not known \citep{agresti_2010_analysis,agresti_1990_categorical,king_a-politically_2007}. The spacing between ``Some High School'' and ``High School Graduate'', for instance, is arguably different than the spacing between ``High School Graduate'' and ``Bachelor's Degree''. It is difficult to measure the distances between the categories, so it is often ignored in academic usage \citep{king_designing_1994,fox_applied_2015}.  Blocking inadequately on such an important variable misrepresents the data and can distort the results, which puts any insights gained from an experiment in doubt. I propose a method that increases precision and removes this doubt.



\subsection{Objectives}

\textit{Hypothesis:} Researchers misuse blocking on education by making it arbitrarily numeric.\par 
\noindent \textit{Goal:} Develop a method to block on education appropriately.

\subsection{Methodology}

I propose an ordered probit model threshold approach to estimate an underlying latent continuous structure underneath education. The re-estimated data-driven categories can then be used for blocking. In practice, I train the following linear model on the 2016 ANES data, one of the most respected and recognized externally and internally valid data sets:

\vspace{-0.6cm}

\begin{equation}
Education \sim Gender + Race + Age + Income + Occupation + Party ID
\end{equation}

\vspace{-0.2cm}

In this model, I regress education on meaningful explanatory variables with the ordered probit function to create numerical thresholds. These thresholds partition education into regions corresponding to its categories and bin the data between these thresholds according to the explanatory variables. These binned cases then determine which of the original categories make sense given the underlying latent continuous structure. The result is a non-arbitrary re-estimated number of categories. Because of their data-driven justification, these categories can then be safely used for blocking. I demonstrate the benefits of this method with several Monte Carlo simulations. Simulations are crucial here as they allow comparison to the `true' results, which is not possible with actual data. They show that the re-estimated ordered probit categories produce analytical results that are closer to the `truth' than the original ANES categories.
 
 
 

\section{A New Method to Impute Missing Survey Data with Ordinal Variables} 

\subsection{Background}

Missing data are ubiquitous in surveys \citep{allison_2002_missing,raghunathan_2016_missing}. Respondents frequently refuse to answer questions, select ``Don't Know'' as a response option, or drop out during the response collection process \citep{honaker_2010_what}. Missing data pose a big problem for researchers because data can typically not be analyzed with statistical software if they contain missing values \citep{little_2002_statistical,molenberghs_2007_missing}. 

Scholars have developed several general ways to treat missing data. These range from deleting all observations with missing data (listwise deletion) over randomly drawing a `similar' respondent to provide a fill-in value for a missing slot (hot decking) to estimating missing values from conditional distributions (multiple imputation) \citep{rubin_1976_inference,king_2001_analyzing,fay_1996_alternative}. Listwise deletion has been shown to induce bias with political data, and hot decking does not reflect statistical uncertainty in the filled-in values since there is only one draw \citep{kroh_2006_taking,gill_2013_bayesian,rees_1997_methods}. While multiple implementation has become and remains the state of the art in missing data management, it is not necessarily always suitable for all types of variables. Multiple hot deck imputation, an improvement over generic multiple imputation, solves this for non-granular discrete data \citep{gill_2012_have, reilly_1993_data}. However, the underlying algorithm assumes even distances between categories in discrete data, which makes it unsuitable for ordinal variables. I propose a method designed to impute missing data specifically from ordinal variables that fills this gap in multiple hot deck imputation. 




\subsection{Objectives}

\textit{Hypothesis:} Current methods to treat missing survey data are unsuitable for ordinal variables. \par 
\noindent \textit{Goal:} Develop a method to impute missing data specifically from ordinal variables.


\subsection{Methodology}

Multiple hot deck imputation uses draws of values from the variable with the missing values (hot decking) to impute them distributionally (multiple imputation) and estimate affinity scores. This score measures how close other respondents are to the one with the missing value. `Closeness' is measured as the distance between respondents in the variables that do not contain missing values. This is best illustrated with simplified data shown in Table \ref{affscore}.

\begin{table}[H]
  \centering
  \singlespacing
  \begin{tabular}{lccccc}
  \bottomrule 
  \midrule
  Respondent & Age & Party ID & Education & Income & Gender\\
  \hline
  A & 25 & Republican & High School Graduate & \$40-50,000 & Male \\
  B & 40 & NA & Some High School &  \$30-40,000 & Female\\
  C & 30 & Democrat & Bachelor's Degree &  \$60-70,000 & Female\\
  \bottomrule 
  \end{tabular}
  \caption{Illustrative Data}
  \label{affscore}
\end{table}

Respondent B shows missing data for party ID. To impute a fill-in value, we look at how close respondents A and C are to B in terms of age, education, income, and gender. C is closer to B in terms of age and they share the same gender. A is closer to B on education and income. Multiple hot deck imputation measures these distances and estimates affinity scores for respondents A and C. B then receives the party ID fill-in value from whichever respondent has the higher score. The algorithm building the affinity score, however, assumes evenly spaced distances between categories. This is the case for age, income, and gender, but not for education, since education is an ordinal variable. Applying multiple hot deck imputation here would misrepresent the data.

Instead, I propose a weighted distance solution with the estimated numeric thresholds from the ordered probit model approach in chapter 1 to measure the distances between the categories and calculate the affinity score. I demonstrate the benefits of this method with several Monte Carlo simulations. As in chapter 1, simulations are crucial as they allow comparison to the `true' results, which is not possible with actual data. They show that weighted distance multiple hot decking imputation outperforms current general missing data techniques for ordinal variables.


\section{Morality vs. Self-Interest in Public Opinion Framing} 

Following the traditional structure of methods dissertations, this chapter applies the methods developed in chapters 1 and 2. I do so in an online survey experiment on political framing. 

We learn about politics through articles, speeches, commercials, and social media. Mediated communication possesses tremendous potential influence on our perception of political issues and thus public opinion \citep{iyengar_framing_1996,kam_risk_2010}. Public opinion is generally categorized into three parts: The material self-interests that we see at stake, sympathies and resentments we feel toward social groups, and commitment to principles and morals \citep{kinder_1998_opinion}. Each of these parts can be shaped by political actors. One of the tools to do so is framing: Framing is the practice of presenting an issue to affect the way people see it \citep{aaroe_investigating_2011,druckman_evaluating_2001}. A variety of frames substantively influence how people view and think about issues \citep{entman_projections_2004,slothuus_political_2010,sniderman_structure_2004,chong_framing_2007-1}. Among these are frames focusing on self-interest and frames focusing on morals.

Morals embody what we feel is `good' and `right'; what everybody should do. They are essential to how we make sense of the world around us and inform our perception of how other people should behave \citep{frank_whats_2005,mooney_public_2001,zaller_nature_1992,haidt_moral_2003}. Self-interest in the narrow sense, on the other hand, concerns aspects that only advance our own personal interests but do not extend to other people. Their sole focus is what is good for us, and just us \citep{brewer_value_2001,skitka_psychology_2010,caprara_2006_personality,converse_nature_1964}. 


While the influence of moral and self-interest frames on public opinion formation has been documented, to my knowledge, these central categories of public opinion formation have not been tested in juxtaposition in experimental research. I propose to do so in an online survey experiment where I will randomly assign moral and self-interest frames to respondents. This will provide valuable insights into the criteria respondents weigh more strongly when they consider political issues and form their opinions. Based on previous research that finds people leaning heavily on values when making political decisions \citep{brader_2008_what,white_2003_values,valentino_2008_worried,transue_2007_identity,stoker_2001_political}, I hypothesize that moral frames elicit larger effects.


%\begin{table}[H]
%\singlespacing
%\centering
%\resizebox{\textwidth}{!}{
%\begin{tabular}{>{\itshape}l ll >{\itshape}l l}
%\bottomrule 
%\midrule
%\multicolumn{2}{c}{\textbf{Positive}} & & \multicolumn{2}{c}{\textbf{Negative}}\\
%\cmidrule{1-2}
%\cmidrule{4-5}
%Care & Cherishing, protecting others & & Harm & Hurting others\\
%Fairness & Rendering justice by shared rules & & Cheating & Flouting justice, shared rules\\
%Loyalty & Standing with your group & & Betrayal & Opposing your group\\
%Respect & Submitting to tradition, authority & & Subversion & Resisting tradition, authority\\
%Sanctity & Repulsion at disgust & & Degradation & Enjoyment of disgust \\
%\bottomrule
%\multicolumn{5}{l}{\footnotesize{Based on Haidt (2012). Positive and negative foundations are conceptual opposites.}} \\
%\end{tabular}}
%\caption{Foundations of Moral Arguments}
%\label{framing-foundations}
%\end{table} 



\subsection{Objectives}

\textit{Hypothesis:} Moral frames achieve larger framing effects than self-interest frames.\par
\noindent \textit{Goal:} Investigate the importance of moral and self-interest frames in influencing public opinion in direct juxtaposition.



\subsection{Data}

I design a questionnaire on several political issues (climate change, abortion, takings, statehood of DC) that collects demographic information and applies several treatments in the form of frames. Each treatment consists of a moral frame and an opposing self-interest frame. All frames will be designed on the basis of political psychology literature and pre-tested on MTurk. The questionnaire is hosted online for a nationally representative sample of 2,331 respondents recruited through Lucid. Lucid  has been shown to perform well on a national scale in survey experiments \citep{coppock_2019_validating}. 

I apply my ordered probit method from chapter 1 by blocking on education. One half of the sample is blocked with arbitrary numeric values, while the other uses my method. Subsequent ordered probit regression on an ordinal response variable on a 5-point Likert scale, ranging from ``Strongly oppose'' to ``Strongly support'', shows the differences in performance. Since there currently is no way to block online, I created an online survey environment based on statistical software code to implement blocking. Missing data in the form of ``Don't Know'' or ``Refused'' are imputed separately with general methods and my specific ordinal variable adaptation of multiple hot deck imputation from chapter 2. Visual displays of the resulting variable distributions show the differences in performance. Substantively, the regression results provide insights into the importance of morals and self-interest in political framing and in political messaging overall. 

\section*{Implications and Limitations}

Education is the most important predictor of political behavior in political science. It is crucial that we measure and use this variable correctly to obtain results that reflect the true data structure. As an ordinal variable, education contains special characteristics: Its categories are ordered, but unevenly spaced. We need modern statistical methods to fully utilize all this information contained in this variable. So far, this aspect has been largely ignored in the literature. If we want to know what people think and how they act, we need to make sure our measurements are as good as they can possibly be. My dissertation outlines two new methods that contribute to this undertaking. They significantly improve how we handle ordinal variables in surveys and survey experiments in political science and thus increase precision when we analyze public opinion.

Whether my methods are suitable in a specific survey experiment depends on the situation. For a survey experiment with a very large sample and few treatment groups, there is likely little need for a special ordered probit blocking method. Simple randomization does the job here. Similarly, if survey results do not contain important ordinal predictor variables, my method to impute missing data from ordinal variables is not applicable. Overall, my dissertation nonetheless adds two important new tools to the empirical political scientist's toolbox to choose from. 


\section*{Timetable for Completion}

\vspace{-0.8cm}

\begin{table}[H]
\singlespacing
\renewcommand\arraystretch{1.4}\arrayrulecolor{NavyBlue}
\captionsetup{singlelinecheck=false, labelfont=sc, labelsep=quad}
\caption*{July 2020 -- May 2020}\vskip -1.5ex
\begin{tabular}{@{\,}r <{\hskip 2pt} !{\foo} >{\raggedright\arraybackslash}p{13.3cm}}
\toprule
\addlinespace[1.5ex]
\textit{July} & Conduct and write-up final simulations \\
\textit{September} & Design and test survey questionnaire \\
\textit{October} & Field survey experiment \\
\textit{November} & Analyze and write-up results \\
\textit{January} & Advanced draft of complete dissertation  \\
\textit{February / March} & Final draft of complete dissertation  \\
\textit{March / April} & Defend dissertation \\
\end{tabular}
\end{table}

\clearpage

\singlespacing

\bibliography{/Users/simonheuberger/Insync/bib-files/all-together.bib}
\bibliographystyle{chicago}



\clearpage

\begin{center}
\LARGE
CV \\
\end{center}

\begin{flushleft}

\section*{Education}

\begin{description}
\item[{\sc Ph.D.}, Political Science]\hspace{-0.175cm}. American University,
Washington, D.C., 2021 (expected) \par
Fields: American Politics, Quantitative Methods, Comparative Politics \\ 
Dissertation: ``What the People Think: Advances in Public Opinion Measurement Using Ordinal Variables''\\
Dissertation Chair: Jeff Gill\\
\end{description}

\begin{description}
\item[{\sc M.A.}, Social and Political Thought]\hspace{-0.175cm}, \textit{magna cum laude}. University of Warwick, UK, 2014 \par
Dissertation: ``Fighting `Socialism': The Koch Brothers, the Tea Party, and Obamacare'' (15,000 words), \textit{magna cum laude} \\
\end{description}

\begin{description}
\item[{\sc Magister}, English Linguistics, Communication.] University of Munich, Germany, 2011 \par
Dissertation: ``From Candidate to President: The Use of Metaphors and Pronouns in Speeches by Barack Obama'' (40,000 words), \textit{magna cum laude} \\
\end{description}


\section*{Publications}
\begin{enumerate}[leftmargin=!,labelindent=20pt,itemindent=-20pt]
\item[] 2020. ``Bayesian Modeling and Inference: A Postmodern Perspective" with Jeff Gill. In Luigi Curini and Robert J. Franzese, Jr. (editors), \textit{The SAGE Handbook of Research Methods in Political Science and International Relations}, p. 961-984.  SAGE (\href{https://www.simonheuberger.com/files/research/handbook_bayes.pdf}{Paper}, \href{https://uk.sagepub.com/en-gb/eur/the-sage-handbook-of-research-methods-in-political-science-and-international-relations/book262875}{SAGE}).
\vspace{-0.27cm}
\item[] 2019. ``Insufficiencies in Data Material: A Replication Analysis of Muchlinski, Siroky, He and Kocher (2016)". \textit{Political Analysis} 27 (1), p. 114-118 (\href{https://www.simonheuberger.com/files/research/2019_pa_insufficiencies_in_data_material.pdf}{Paper}, \href{https://www.cambridge.org/core/journals/political-analysis/article/insufficiencies-in-data-material-a-replication-analysis-of-muchlinski-siroky-he-and-kocher-2016/DCFFD3F8F23604794ABE615F10C42FA4}{DOI}).
\end{enumerate}


\section*{Teaching Experience}



\textbf{Adjunct Instructor}

\vspace{-0.3cm}
\begin{itemize}
\item[] Political Action and Public Policy (Online, \href{https://www.simonheuberger.com/files/teaching/syll_summer_2019_102.pdf}{Syllabus}) \hfill{Summer 2019}\\
Politics in the U.S. (Online, \href{https://www.simonheuberger.com/files/teaching/syll_summer_2019_110.pdf}{Syllabus}) \hfill{Summer 2019}\\
Introduction to Political Research (\texttt{R}) (\href{https://www.simonheuberger.com/files/teaching/syll_fall_2018_310.pdf}{2018 Syllabus}, \href{https://www.simonheuberger.com/files/teaching/evals_fall_2018_310.pdf}{Evaluations}) \hfill{Fall 2018}
\end{itemize}


\textbf{Teaching Assistant}

\vspace{-0.3cm}
\begin{itemize}
\item[] Institute for Data Science (\texttt{R}, \texttt{Python}, \texttt{JS}, \texttt{SQL}), Ryan T. Moore (\href{https://www.american.edu/spa/data-science/upload/syl-2020.pdf}{Syllabus}) \hfill{Winter 2019/20}\\

Bayesian Statistics for Social and Biomedical Sciences (\texttt{R}), Jeff Gill (\href{http://jeffgill.org/classes/american-university-statistics-618spa-696-every-fall-bayesian-statistics-social-and}{Syllabus}) \hfill{Fall 2018}\\

Introduction to Political Research (\texttt{R}), Ryan T. Moore (\href{https://www.simonheuberger.com/files/teaching/syll_spring_2017_310.pdf}{Syllabus}, \href{https://www.simonheuberger.com/files/teaching/intro_using_r.pdf}{Guide to R}) \hfill{Spring 2017}\\

Introduction to Political Research (\texttt{Stata}), Jan E. Leighley (\href{https://www.simonheuberger.com/files/teaching/syll_summer_2016_310.pdf}{Syllabus})\hfill{Summer 2016}
\end{itemize}



\section*{Research Experience}

\textbf{Editorial Assistant} \hfill{2017-Present}

\begin{tabular*}{1\textwidth}{@{\extracolsep{\fill}}l}
Political Analysis Journal (Jeff Gill, R. Michael Alvarez, Jonathan Katz)
\end{tabular*}
\vspace{-0.6cm}

\begin{itemize}
\item[] Data Replication of Journal Submissions \\
Code Debugging and Quality Control in \texttt{R}, \texttt{Python}, \texttt{Stata}\\
Dataverse Organizational Management
 
\end{itemize}



\textbf{Research Assistant} \hfill{2015-Present}

% making two tables gives better vertical distance

\begin{tabular*}{1\textwidth}{@{\extracolsep{\fill}}l}
Matthew Wright, Ryan T. Moore, Jeff Gill
\end{tabular*}
\vspace{-0.6cm}

\begin{itemize}
\item[] Strength and Effectiveness of Framing Measures in Political Behavior \\
Immigration as a Focal Aspect of the 2016 Presidential Race\\
List Experiments on Social Desirability Bias\\
Development of Aptitude Tests for Data Scientist Positions at The Lab @ DC\\
Enhancement of {\tt R} Package {\tt blockTools}\\
Establishment of American University Center for Data Science Dataverse (\href{https://dataverse.harvard.edu/dataverse/americanu_cds}{Dataverse})\\
MCMC Dynamics Measuring State Ideology with Spatial Variance Co-Matrices
\end{itemize}




\section*{Data Analysis}
\begin{enumerate}[leftmargin=!,labelindent=20pt,itemindent=-20pt]
\item[] {\tt Eagledown}: {\tt R} Package to Write a PhD Thesis at AU in {\tt R Markdown} (\href{https://github.com/SimonHeuberger/eagledown}{GitHub})
\vspace{-0.27cm}
\item[] {\tt GLMpack}: {\tt R} Package to Accompany \textit{Generalized Linear Models: A Unified Approach} (2nd ed.) by Jeff Gill and Michelle Torres (\href{https://cran.r-project.org/web/packages/GLMpack/index.html}{CRAN})
\vspace{-0.27cm}
\item[] {\tt BlockExperiments}: {\tt R} Package to Use Ordinal Variables for Blocking (\textit{in development})
\vspace{-0.27cm}
\item[] \texttt{Python}, SQL, AWS, Docker, GitHub, \LaTeX, Qualtrics, \texttt{Stata}, \texttt{SPSS} 
\end{enumerate}




\section*{Awards and Grants}

\begin{enumerate}[leftmargin=!,labelindent=20pt,itemindent=-20pt]
\item[] Gill Family Foundation Scholarship, 2019
\vspace{-0.27cm}
\item[] AAPOR Travel Award, 2019
\vspace{-0.27cm}
\item[] BITSS Scholarship for Research Transparency and Reproducibility Training (RT2), 2018 
\vspace{-0.27cm}
\item[] Office of the Provost Doctoral Student Research Scholarship, 2018
\vspace{-0.27cm}
\item[] APSA Annual Meeting Travel Grant, 2018
\vspace{-0.27cm}
\item[] Prestage-Cook Award, SPSA Annual Meeting, 2018
\vspace{-0.27cm}
\item[] NSF Grant, Presentation at Society for Political Methodology, 2017, 2019
\vspace{-0.27cm}
\item[] Graduate Leadership Council Conference Travel Grant, 2017, 2018, 2019
\vspace{-0.27cm}
\item[] Award from the Vice Provost and Dean of Graduate Studies, ``The Impact of Information and Emotions on Voter Turnout and Civic Engagement,'' with Jan E. Leighley, 2016
\vspace{-0.27cm}
\item[] Department of Government Graduate Research Support, ``Framing Methodologies,'' 2016
\vspace{-0.27cm}
\item[] Graduate Assistantship, 2015-2019
\vspace{-0.27cm}
\item [] German Academic Exchange Service Scholarship for Study Abroad in Australia, 2009
\end{enumerate}




\section*{Other}

\begin{itemize}
\item[] \textbf{Languages}\\
German \textit{(native)} \\
English \textit{(bilingual proficiency)} \\
Spanish \textit{(full proficiency)} \\
French \textit{(full proficiency)} \\
Arabic \textit{(basic proficiency)} \\
\end{itemize}



\pagestyle{myheadings}

\end{flushleft}
 
\end{document}