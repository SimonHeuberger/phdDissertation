\documentclass[10pt,letterpaper,dateno,sigleft]{newlfm}

\usepackage{charter} % Use the Charter font for the document text
\usepackage{graphicx}
\usepackage[letterspace=100]{microtype} % To make the letters of \LaTeX a bit more spaced out (the font style presses words together)
\usepackage{float}
\usepackage{hyperref}
\hypersetup{
    colorlinks = true,
    urlcolor = {blue}
}

\newlfmP{sigsize=50pt} % Slightly decrease the height of the signature field


\makeatletter % These lines remove the black lines
\g@addto@macro{\ps@ltrhead}{%
  \renewcommand{\headrulewidth}{0pt}%
  \renewcommand{\footrulewidth}{0pt}%
}
\g@addto@macro{\ps@othhead}{%
  \renewcommand{\headrulewidth}{0pt}%
  \renewcommand{\footrulewidth}{0pt}%
}
\makeatother

\addrfrom{
\today\\ % Date
Washington, DC
\vspace{-1cm}
}

\addrto{
Gill Family Foundation Scholarship\\
School of Public Affairs, American University
}


\begin{document}

\begin{newlfm}


\vspace{-0.3cm}

Dear Selection Committee,

\vspace{0.15cm}

I am writing to express my interest in the Gill Family Foundation Scholarship. I am a fifth-year Ph.D. candidate in American Politics. I took the American and Comparative comprehensive exams in 2017 as well as the the Methods comprehensive exam in 2018. I am currently writing the first-ever methods dissertation in the Government Department at American University, which I plan to defend in the summer of 2020. 

I started at AU without prior knowledge in statistics. The methods courses in the Conduct of Inquiry sequence introduced me to this world, and I quickly discovered my passion for data analysis. Thanks to the four courses, each of which I completed with an A, I now possess a wide knowledge of the proper usage of and assumptions behind a wide variety of statistical modeling strategies. This knowledge greatly informs my approaches towards analyzing data and also significantly enhances my understanding of quantitative literature. These classes taught me the use of the statistical software packages {\tt R} (Conduct I and II) and {\tt Stata} (Conduct III and IV), as well as the typesetting language {\lsstyle \LaTeX}. Through my own self-teaching thereafter, I am now also fluent in {\tt Python}, {\tt R Markdown}, and {\tt SQL}.


As part of my Graduate Assistantship, I worked with Prof. Ryan T. Moore and Prof. Jeff Gill. As my instructor in Conduct I and II, Prof. Moore's influence on my proficiency in statistics and quantitative methods cannot be overstated. Under his tutorship, I also had the opportunity to teach undergraduates about political science methodology in \textit{GOVT-310} in 2017, which inspired me to teach this course myself as an adjunct instructor in 2018. Since August 2017, I have been working with Prof. Gill. He is currently the editor-in-chief of the journal \textit{Political Analysis}, where I am in charge of replicating and debugging all data material for conditionally accepted manuscripts. Due to the intensive computational nature of most material, I taught myself the use of elastic cloud computing (EC2) instances on Amazon Web Services and self-contained data capsules on Code Ocean. Prof. Gill and I are currently implementing the switch from Dataverse storage towards Code Ocean; the first of its kind in the social sciences. My replication work has led to a single-authored publication in \textit{Political Analysis}, where I dissect insufficient published replication material. 
 

The Conduct series, teaching statistical methods with Prof. Moore, and conducting advanced data analysis with Prof. Gill have inspired me to write a quantitative methods dissertation. My interests are focused on survey experiments and data measurement, particularly the use of ordinal variables, i.e. variables with ordered categories where the spacing between each category is uneven and not known. In the analysis of survey experiments, ordinal categories are often assigned equally spaced numeric values, thereby making them effectively nominal. This is problematic because the assumption of equally spaced distances is not supported, which in turn misrepresents the data and can distort experimental results. My dissertation develops and applies new methods to utilize ordinal variables in survey experiments without relying on this unwarranted assumption. \\
The first method concerns the assignment of respondents into treatment groups. I apply an ordered probit model to estimate a latent underlying continuous variable, which approximates the true spacing between the variable categories and outputs the variable categories we should be using. The results are then used to block respondents into treatment groups. \\
The second method concerns missing data. I apply a similar ordered probit model to estimate the true underlying distances between ordinal variable categories. Together with other variables, these distances are used to estimate affinity scores for observations without missing values. Missing values are then multiply imputed with donor observations with the lowest score. 
Both methods take into account and utilize the unique nature of ordinal variables. They are extensively simulated and then applied in an original online survey experiment on political framing.\\
Framing research has established that a variety of frames substantively influence how people view and think about political issues, but we do not know why these frames elicit these effects. A major challenge for framing research thus concerns the identification of factors that make a frame strong. Based on moralization theory, I investigate whether moral arguments are part of what makes a political frame strong. I do so in an online survey experiment with 2,300 respondents through Lucid. I was able to secure \$5,000 in funding through the Office of the Provost to conduct this experiment.


This project has evolved considerably since the prospectus defense. As a result, the enclosed prospectus is somewhat dated. The methods chapters in particular have changed significantly.

\vspace{0.15cm}
I thank you for taking the time to consider my application.

\vspace{0.15cm}

Sincerely, 
\newline \newline \includegraphics[width=4cm, height=2cm]{/Users/simonheuberger/Dropbox/work/signature.jpg}

Simon Heuberger


\end{newlfm}

\end{document}