\documentclass[12pt]{article}

\input{\string~"/Insync/templates/preamble2"}

\title{Introduction}

\date{}

\begin{document}

\maketitle

%\section*{Feedback Ryan}
%	\begin{coi}
%		\item p ii. Add a short description of the findings in each case
%		\item Intro has a lot of "behavior", but opinion and attitudes are more what survey research has been about
%		\item p 3-4. You deal here with the case of making ordinals interval. You don't address the other most common solution (and its drawbacks) -- using category indicators -- until p. 17. Add that idea here, too
%		\item p 4. It's balance on the potential outcomes that we actually care about; balance on predictive covariates is just a way to get there, hopefully, since we don't know the outcomes. Same issue on p. 6. The chapter starts to get this right on page 9. Communicate clearly throughout the thesis about this important idea
%		\item p 5. What's unsuitable about standard multiple imputation? Distributional assumptions?
%		\item State all findings here briefly
%	\end{coi}

%\section*{Feedback Liz}
%	\begin{coi}
%		\item Abstract: For ch 4, remark also about how the methods innovations turn out
%	\end{coi}








\end{document}


