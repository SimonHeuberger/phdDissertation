\input{/Users/simonheuberger/Dropbox/work/latex/preamble}

\title{Framing}

\date{\today}

\begin{document}

\maketitle

%\section*{\st{Prospectus Revisions}}
%	\begin{coi}
%		\item \st{Ryan: Replace focus group with online MTurk. That can give me the same insights for a lot less money and effort}
%		\item \st{Proofread everything}
%	\end{coi}

%\section*{Pre-Polls}
%	\begin{coi}
%		\item I need to know how many pre-polls I'm running and how much money I should allocate for that (since it all needs to be within the Provost money)
%		\item I need to know that now because I need to tell Lucid how much much I'm paying them by mid-March
%		\item The current setup from the proposal is meta-analysis, three simple polls, experiment:
%			\begin{coi}
%				\item The meta-analysis gives me a list of all statistically significant frames
%				\item Poll \#1 asks respondents to rate how moral each argument in each of those frames is
%				\item Poll \#2 asks people what they consider moral arguments to be. I use the insights from poll \#2 to design frames for each of the `missing' frames in previous experiments (the idea being that no experiment so far has tested for positive/negative moral/amoral frames and there thus will be gaps)
%				\item Poll \#3 tests how moral/amoral my designed frames are
%				\item Then finally comes the survey experiment
%			\end{coi}
%		\item I have \$4,830. If I did one simple one for \$0.10 at 200 respondents, that would cost me \$28. So it would be \$84 for three simple polls with 200 respondents each, which leaves me with \$4,746. My R method online with Lucid costs \$2 each, so that would make 2,373 respondents for the experiment. That would leave 474 respondents per treatment group
%		\item Money-wise, that would work out. Now I need to know whether the proposal setup (meta-analysis, pre-polls) is feasible to do in a realistic time frame, with me teaching in both terms
%		\item Very good point from Ryan: He totally sees the point of polls 1 and 3, but not really poll 2. Removing 2 would obviously change the financial numbers -- but Rune Slothuus for instance said my idea for the open-ended poll about what makes frames moral is good
%		\item Jeff and Matt agree: Ditch poll \#2
%		\item Go with 300 respondents for each of the remaining pre-polls. To make sure people pay more attention to the frames, pay \$0.20 each. That comes out to \$168 for both pre-polls, which leaves \$4,662 for Lucid, which gives me 2,331 respondents, which means 466 respondents per treatment group
%	\end{coi}

\section*{General}
	\begin{coi}
		\item Design moral and unmoral frames and pretest both on MTurk, then use the `proven' frames for the survey experiment. This is a chapter on framing where I apply both methods in the experiment, analyze the methods' performances, and analyze the substantive results in terms of moral frames (the latter is where I produce/add some nuance in our understanding of framing)
		\item Using the experiment as the application for both papers
			\begin{coi}
				\item Paper I: One half of the sample gets the ordinal probit education categories, the other gets the original ones without giving respondents \texttt{Don't Know/Refuse} options. Compare the differences between the ordered probit and the original results whilst knowing which one is closer to the truth based on the simulation results
				\item Paper II: Use the resulting completely observed data, introduce random missing data, and show how the ordinal affinity score method performs on ordinal variables compared with other method	. Assess the performance of the ordinal affinity score method because we know the true values of the completely observed data
			\end{coi}
	\end{coi}	
		

\section*{Order}
	\begin{coi}
		\item Pre-poll \#1
		\item Design gap-filling frames
		\item Pre-poll \#2
		\item Design questionnaire
		\item New IRB approval
		\item Pre-analysis plan and pre-registration
		\item Field experiment
		\item Analysis of the results
	\end{coi}

\section*{Pre-poll \#1}
	\begin{coi}
		\item After paper II is done
		\item 300 respondents, \$0.20 each
	\end{coi}

\section*{Design gap-filling frames}
	\begin{coi}
		\item After pre-poll \#1 is done
	\end{coi}

\section*{Pre-poll \#2}
	\begin{coi}
		\item After gap-filling frames are designed
		\item 300 respondents, \$0.20 each
	\end{coi}

	
\section*{Design questionnaire}
	\begin{coi}
		\item After pre-poll \#2 is done
		\item Feedback from MPMC
			\begin{coi}
				\item The treatment is too weak, not powerful enough
					\begin{coi}
						\item You have to go big, hit them with a sledgehammer
						\item Have them read newspaper articles, bigger things, longer, more; not just one sentence
					\end{coi}
				\item Morality is in stories, pragmatism is in numbers -- incorporate that into frames
			\end{coi}
		\item Feedback from Rune Slothuus, APSA, 9/2/2018
			\begin{coi}
				\item He says what I do is emphasis framing (other papers use different versions)
				\item Emphasis framing: Communication that puts emphasis on a salient aspect of an issue
				\item Leading opinion in a certain direction is different from providing an argument
				\item Confounding framing and arguments could be troublesome (Slothuus wrote a paper on this)
				\item Important step: What is my concept of framing? The opposing frame weren't really frames here, but arguments
			\end{coi}
	\end{coi}

\section*{New IRB approval}
	\begin{coi}
		\item After questionnaire is designed
	\end{coi}

\section*{Pre-analysis plan and pre-registration (from RT2 training)}
	\begin{coi}
		\item After IRB approval is done
		\item Articles about invisible null fundings in research community: Franco et al. 2014, Rosenthal 1979
		\item Read Miguel et al. 2014 ``Promoting transparency in social science research" (in the BITSS /readings folder)
		\item Just because you have a pre-analysis plan doesn't mean that's all you can do with your data (and it doesn't mean exploratory analysis is now forbidden). Just be careful why you're doing it. If it's to improve the p-value, obviously don't do it. If there is a good theoretical motivation that you can defend, then it might be a different story
		\item Pre-registration can help with publication bias and improve meta-analysis
	\end{coi}

\section*{Field experiment}
	\begin{coi}
		\item After pre-analysis and pre-registration are done
		\item Before I launch anything:
			\begin{coi}
				\item What variables am I blocking on?
				\item Correct OPM education categories?
				\item Be INCREDIBLY careful with any randomization of response options. Many things in the code use the corresponding number for respondents' response selections, not the actual words, so messing with the order is a very delicate thing
				\item Anything I might have missed somewhere?
			\end{coi}
		\item Lucid will wait for my go-ahead to launch the actual project whenever I'm ready. Then we'll schedule a kick-off call and talk about project details/logistics
	\end{coi}
	
\section*{Analysis of the results}
	\begin{coi}
		\item Most straightforward is to use the method I use to block
		\item But I also have lots of good resources how to analyze ordinal variables as EVs, instead of turning them into intervals for a normal regression, including a Bayesian way
	\end{coi}



\end{document}
