\documentclass[12pt]{article}

\input{\string~"/Insync/templates/preamble2"}

\title{Defense Feedback}

\date{\today}

\begin{document}

\maketitle

\section*{Jeff}
	\begin{coi}
		\item hot.deck is non-parametric, which means it doesn't care about distances. Change my comments on that
		\item Include a little more on Type I/II errors -- which one does it increase more than the other? It can't be symmetric, i.e. that both happen equally. Explore that a bit more. Jeff: It's going to be that we're underestimating Type I errors. Type II errors probably won't be affected
		\item Include a paragraph on the notion that different ordinal categories mean different things to different people, i.e. the subjective perception of 5-point scales isn't the same across people, and whether/how that influences measuring distances between categories
		\item hd.ord does better, i.e. more on par with the other methods, for MNAR because the methods are all equally bad here. None of them were designed for it, so none of them do well. Make sure to include this point when I adjust the conclusions to make them more positive
		\item Why did I not get stronger results in the blocking chapter? Why not the results I wanted? (All the other chapters had some form of 'why' in them, so put something here as well)
	\end{coi}

\section*{Ryan}
	\begin{coi}
		\item On p. 33, another example of missing data methods making a political difference is available at \href{http://www.ryantmoore.org/files/papers/wlidd.pdf}{http://www.ryantmoore.org/files/papers/wlidd.pdf}
		\item p. 39. For packages that do hot decking, see \\\href{https://cran.r-project.org/web/views/MissingData.html}{https://cran.r-project.org/web/views/MissingData.html}
		\item Equation (2.7), the one with the model for the ANES: The predictor variables here predict education. Education is on the left side here, but then becomes the right side for subsequent analysis. Are there potential problems with this in terms of bias?
		\item Figure 2.4, linear predictor distribution: What if this returns a distribution with big gaps between categories left and right? How would we continue with category reassignment in such a case? Would we still give the new categories sequential numerical values?
		\item Is there any non-response in the Lucid data? If not, is that because they were required to be completes? (In a normal (?) survey, we'd have some non-response.) Maybe just a little clarification here
	\end{coi}

\section*{Betty}
	\begin{coi}
		\item v2: Figure 2.1: Explain why the sample sizes differ between rows, i.e. 14 for 2 groups, 18 for 3, 25 for 5 etc.
		\item v2: p. 53, 54 clarify ``number of imputations", ``1,000"
		\item v2: p. 54. 57 clarify how interest is included in the polr treatment (confusingly worded)
		\item v3: bottom p. 90 was there no actual missing data in the experiment?
		\item v4: On p. 12 the steps for the estimation process with blocking has step 8 repeating steps 1 to 4. Are only 1 to 4 repeated? Or should 1 to 7 be repeated?
	\end{coi}

\section*{Mike}
	\begin{coi}
		\item Conclusions are too negative, there are some cases when my method might work. There might be some substantive situations where the ordinal responses are probably mapping well from the latent variable, while in other situations that mapping may not be straightforward. Adjust diss to sound more like the presentation here. Point conclusion in a more positive direction, i.e. that there is the need for further exploration of this problem to get us to a point where we might have a good diagnostic test for when we should assume the ordinal responses might map well to a latent variable and when they might not
	\end{coi}







\end{document}


