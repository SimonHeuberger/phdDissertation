
%%%%%%%%%%%%%%%%%%%%%%%%%%%%%%%%%%%%%%%%%%%%%%%%%%%%%%%%%%%%%%%%%%%%%%%%%%%%%%%%%%%%%%%%%
\documentclass[12pt]{article}
%\usepackage[dvips]{epsfig}
\usepackage{pstricks,pst-node,pst-tree}
\usepackage[dvips]{graphicx}
\usepackage{floatflt,epsfig}
\usepackage{amsfonts, amssymb, amsmath}
\usepackage{portland}
\usepackage{setspace}
\usepackage{enumerate}
\usepackage{color}
\usepackage{esint}
\usepackage{vmargin}
        \setpapersize{USletter}
        \setmarginsrb{1.0in}{1.0in}{1.0in}{0.3in}{0in}{0in}{0in}{0in}
\usepackage{fancyhdr}
        \pagestyle{fancyplain}
        \setlength{\footskip}{0pt}
                \setlength{\headsep}{40pt}
                \setlength{\topmargin}{28pt}
        \addtolength{\headwidth}{\marginparsep}
        \renewcommand{\headrulewidth}{0pt}
        \lhead{\fancyplain{}{\bfseries \small%
		    \textcolor{Mygrey}{Center for Behavioral Neuroscience, February 27, 2019 [\thepage] }} }
		\cfoot{}
	\usepackage{multirow}
	\usepackage{fancyvrb}
	\usepackage{verbatim}

\def\bupsilon{\boldsymbol{\Upsilon}}
\def\aet{\mathbf{A}\boldsymbol{\eta}}
\def\bA{\mathbf{A}}
\def\bC{\bf C}
\def\bI{\mathbf{I}}
\def\bXal{\X^{(\ell)}}
\def\ba{\mathbf{A}}
\def\bbeta{\boldsymbol{\beta}}
\def\EPS{\boldsymbol{\epsilon}}
\def\boeta{\boldsymbol{\eta}}
\def\eti{(\mathbf{A}\boldsymbol{\eta})_i}
\def\etki{(\mathbf{A}_{k^{(t)}}\boldsymbol{\eta}_{k^{(t)}})_i}
\def\meta{\boldsymbol{\mu_\eta}}
\def\mk{m^k\mathcal{L}_k}
\def\mxb{\X_i\boldsymbol{\beta}}
\def\mxlb{\X_l\boldsymbol{\beta}}
\def\pa{\partial}
\def\psci{\psi_{C_i}}
\def\psk{\psi_{S_k}}
\def\sqp{\sqrt{2\pi}}
\def\sumi{\sum_{i=1}^n}
\def\sumk{\sum_{k=1}^n}
\def\ud{\textrm{d}}
\newcommand{\BigPhi}{ \text{\parbox{0.18in}{\begin{huge}$\Phi\;$\end{huge}}} }
\newcommand{\ALPHA}{\boldsymbol{\alpha}}
\newcommand{\Amat}{\mathbf{A}}
\newcommand{\BETA}{\boldsymbol{\beta}}
\newcommand{\B}{\boldsymbol{\beta}}
\newcommand{\DELTA}{\boldsymbol{\delta}}
\newcommand{\PP}{\boldsymbol{\psi}}
\newcommand{\PSI}{\boldsymbol{\psi}}
\newcommand{\TAU}{\boldsymbol{\tau}}
\newcommand{\THETA}{\boldsymbol{\theta}}
\newcommand{\T}{{\boldsymbol{\theta}}}
\newcommand{\BigT}{{\boldsymbol{\Theta}}}
\newcommand{\Var}{\text{Var}}
\newcommand{\Y}{\mathbf{Y}}
\newcommand{\Z}{\mathbf{Z}}
\newcommand{\bSig}{\boldsymbol{\Sigma}}
\newcommand{\bS}{{\mathbf{S}}}
\newcommand{\bU}{{\bf U}}
\newcommand{\bX}{{\bf X}}
\newcommand{\bY}{{\bf Y}}
\newcommand{\bc}{{\bf c}}
\newcommand{\bdelta}{\boldsymbol{\delta}}
\newcommand{\beps}{\boldsymbol{\epsilon}}
\newcommand{\blsquote}{\renewcommand{\baselinestretch}{0.9}}
\newcommand{\blsref}{\renewcommand{\baselinestretch}{0.9}}
\newcommand{\blstable}{\renewcommand{\baselinestretch}{1.0}}
\newcommand{\bls}{\renewcommand{\baselinestretch}{1.58}}
\newcommand{\boa}{{\normalfont{\texttt{BOA}\;}}}
\newcommand{\boldeta}{\boldsymbol{\eta}}
\newcommand{\bom}{{\bf m}}
\newcommand{\bq}{{\bf q}}
\newcommand{\bugs}{{\normalfont{\texttt{BUGS}\;}}}
%\newcommand{\bxi}{\mathbf{\xi}_i}
\newcommand{\ee}{\mathbf{e}}
\newcommand{\m}{{\boldsymbol{\mu}}}
\newcommand{\winbugs}{{\normalfont{\texttt{WinBUGS}\;}}}
\newcommand{\z}{\mathbf{z}}
\newcommand{\PHI}{\boldsymbol{\phi}}
\newcommand{\BPHI}{\boldsymbol{\Phi}}
	\newcommand{\etal}{{\it et al. }}
	\newcommand{\range}{\negmedspace:\negmedspace}
	\newcommand{\bh}{\mathbf{b}}
	\newcommand{\code}[1]{\textcolor{MyGreen}{\texttt{#1}}}
	\newcommand{\M}{\boldsymbol{\mu}}
	\newcommand{\SI}{\boldsymbol{\Sigma}}
	\newcommand{\SIinv}{\boldsymbol{\Sigma}^{-1}}
	\newcommand{\bb}{\boldsymbol{\beta}}
	\newcommand{\ga}{\boldsymbol{\gamma}}
	\newcommand{\G}{\boldsymbol{\gamma}}
	\newcommand{\I}{{\mathbf{I}}}
	\newcommand{\ttau}{\color[cmyk]{0.92,0,0.59,0.25}{\tau}}
	\newcommand{\9}{\\[9pt]}
	\newcommand{\R}{\texttt{R}}
	\newcommand{\vo}{\mathbf{v}}
	\newcommand{\ca}{\mathbf{c}}
	\newcommand{\U}{\mathbf{U}}
	\newcommand{\X}{\mathbf{X}}
	\newcommand{\x}{\mathbf{x}}
	\newcommand{\y}{\mathbf{y}}
	\newcommand{\E}{\mathbf{E}}
	\newcommand{\D}{\mathbf{D}}
	\newcommand{\OO}{\mathbf{O}}
	\newcommand{\dd}{\mathbf{d}}
	\newcommand{\half}{\frac{1}{2}}
	\renewcommand{\bibitem}{\vskip 2pt\par\hangindent\parindent\hskip-\parindent}
	\renewcommand{\baselinestretch}{1.00} 
	\renewcommand{\thesection}{\Alph{section}}
	\newcommand{\coda}{{\normalfont{\texttt{CODA}\;}}}
\newcommand{\bgm}{\color{MyMath}}
\newcommand{\egm}{\color{white}}
\DefineVerbatimEnvironment{VM}{Verbatim}{formatcom=\color{MyGreen}}
\renewenvironment{equation}{\bgm\begin{equation*}}{\end{equation*}\egm}

	\makeatletter
	  \renewcommand{\section}{\@startsection
	    {section}%
	    {1}%
	    {0mm}%
	    {0.1\baselineskip}%-0.9
	    {0.8\baselineskip}%0.3
	    %{\newpage \color[cmyk]{0.43,0,0.01,0} \center\LARGE}}% style
	    {\newpage \color[cmyk]{0.43,0,0.01,0} \center\LARGE}}% style
	  \renewcommand{\subsection}{\@startsection
	    {subsection}%
	    {2}%
	    {0mm}%
	    {0.4\baselineskip}%-0.8
	    {0.74\baselineskip\@afterindenttrue}%0.12 0.05
	    {\Large\itshape}}
	\makeatother

\begin{document}\begin{landscape}
\pagecolor{black}

	\newenvironment{ohlist}
	    { \begin{list}{$\blacktriangleright$}{\setlength{\itemsep}{10pt}} }
	    %{ \begin{list}{$\clubsuit$}{\setlength{\itemsep}{20pt}} }
	    { \end{list} }
	\newenvironment{ohlistsmall}
	    { \begin{list}{$\blacktriangleright$}{\setlength{\itemsep}{1pt}} }
	    %{ \begin{list}{$\clubsuit$}{\setlength{\itemsep}{1pt}} }
	    { \end{list} }
	\newenvironment{ohlist2}
	    { \begin{list}{$\triangleright$}{\setlength{\itemsep}{8pt}} }
	    %{ \begin{list}{$\spadesuit$}{\setlength{\itemsep}{8pt}} }
	    { \end{list} }
	\newenvironment{ohlist3}
	    { \begin{list}{$\triangleright$}{\setlength{\itemsep}{1pt}} }
	    %{ \begin{list}{$\spadesuit$}{\setlength{\itemsep}{1pt}} }
	    { \end{list} }
	\definecolor{MyMath}{cmyk}{0.00,0.16,1.00,0.00}
	\definecolor{MyGreen}{rgb}{0,1,0}
	\definecolor{Mygrey}{gray}{0.65}
	\definecolor{MyGold}{cmyk}{0,0.11,0.99,0}
	\definecolor{MyEmph}{cmyk}{0,0.61,0.72,0.07}

\begin{titlepage}
\begin{center}\begin{huge}
\noindent { \textcolor{MyGold}{Pediatric Traumatic Brain Injury: Progress and Plans${}^{\ddagger}$}}\\[0.5in]
\end{huge}
\begin{Large}
\parbox{\linewidth}{  \begin{center} \begin{tabular}{c}
    \textcolor{MyGreen}{JEFF GILL}  \\
    {\bf \textcolor{white}{\em Distinguished Professor, Department of Government} }\\
    {\bf \textcolor{white}{\em Professor, Department Mathematics \& Statistics} }\\
    {\bf \textcolor{white}{\em Member, Center for Behavioral Neuroscience} }\\
    {\bf \textcolor{white}{\em American University} }\\[2.2in]
\end{tabular} \end{center} }
\end{Large} %\normalsize
\end{center}
    {\bf \textcolor{white}{\large $\ddagger$ Supported by Eunice Kennedy Shriver National Institute of Child Health \& Human Development (NIH)
            Grant Number: 1R21HD086784-01A1.} }\\
\pagestyle{empty}
\end{titlepage}

%%%%%%%%%%%%%%%%%%%%%%%%%%%%%%%%%%%%%%%%%%%%%%%%%%%%%%%%%%%%%%%%%%%%%%%%%%%%%%%%%%%%%%%%%%
%%%%%%%%%%%%%%%%%%%%%%%%%%%%%%%%%%%%%%%%%%%%%%%%%%%%%%%%%%%%%%%%%%%%%%%%%%%%%%%%%%%%%%%%%%
\begin{Large}
\color{white}{

\section*{Pediatric Traumatic Brain Injury}
\begin{ohlist}
    \item   Severe TBI remains a leading cause of pediatric death and disability. 
    \item   Motor vehicle accidents, falls, and abusive head trauma constitute the most common etiologies. 
    \item   Pharmacological neuroprotective therapies are \emph{not} available for severe TBI, but guideline-based intensive 
            care can improve outcomes. 
    \item   Guideline-based intensive care recommends avoidance of secondary insults that are consistently associated with 
            abnormal brain metabolism and bad outcome, including intracranial hypertension, hyperventilation, hypoxia, and 
            occasionally hypotension. 
    \item   ICP monitoring is a complex clinical issue: uneven application, nonlinear trajectory, poor predictive power.
    \item   \underline{Today:}
            \begin{ohlist2}
                \item   a brief discussion of a published paper showing the value of guideline based intensive care for PTBI.
                \item   prelimary results from ongoing (funded by NIH) research on biochemical-based analysis
                        intended for evential clinical prescription.
            \end{ohlist2}
\end{ohlist}

\section*{General Measurement Issues for TBI}
\begin{ohlist}
    \item   Individual patient biological measures at discharge are often pseudo-interval (continuous) measures, but we also desire 
            qualitative descriptive measures that summarize patient status.
    \item   Clearly a standard categorical measure with many values would not be helpful here: for a \bgm$0-100$\egm\ scale what would it mean to be
            a \bgm$37$\egm\ versus a \bgm$41$\egm?
    \item   On the other end of the scale, dichotomous outcomes such as lived/died are too blunt to reflect the range of important
            potential outcomes.
    \item   In both of the last cases, it is common for the for the data not to support modeling these outcomes in a regression
            context.
\end{ohlist}

\section*{Implemented Measures}
\begin{ohlist}
    \item   Accordingly, the most useful scales for clinical outcomes have a modest number of informed categories.
    \item   For example, the \textcolor{MyEmph}{Extended Glasgow Outcome Scale} (GOSE) is structured according to:
            \begin{center}
            \begin{tabular}{|c|c|c|}
                \hline
                1   & Death                     & D \\
                2   & Vegetative State            & VS \\
                3   & Lower Severe Disability   & SD- \\
                4   & Upper Severe Disability   & SD+ \\
                5   & Lower Moderate Disability & MD- \\
                6   & Upper Moderate Disability & MD+ \\
                7   & Lower Good Recovery       & GR- \\
                8   & Upper Good Recovery       & GR+ \\
                \hline
            \end{tabular}
            \end{center}
            \vspace{11pt}
            (cf. Wilson JT, Slieker FJ, Legrand V, Murray G, Stocchetti N, Maas AI. Observer variation in the assessment of
            outcome in traumatic brain injury: experience from a multicenter, international randomized clinical trial. Neurosurgery.
            Jul;61(1):123-8; discussion 128-9. 2007.)
    \item   Other similar categorized hospital discharge disposition scales also exist.
\end{ohlist}

\section*{Unhelpful Measures For Statistical Outcomes Assessment}
\begin{ohlist}
    \item   Dichotomous survival, unless measured over time and uncensored.
    \item   From ``ASSESSING A PATIENT’S RISK AT DISCHARGE; USING A TOOL TO IDENTIFY APPROPRIATE POST DISCHARGE RESOURCES''
            (eq.Health study,\\ \texttt{http://www.cfmc.org/integratingcare/files/AssessingRiskatDischargeRev4-11.pdf}):
            \begin{center}
            \begin{tabular}{|c|c|}
                \hline
                Home With No Resources      & 45\% \\
                Home Health                 & 23\% \\
                Hospice                     & 1\% \\
                Skilled Care                & 11\% \\
                Long Term Acute Care/Rehab  & 9\% \\
                Other                       & 11\% \\
                \hline
            \end{tabular}
            \end{center}
            \vspace{11pt}
    \item   Status codes, such as the Patient Discharge Status Codes on BCBS UB04 claim form, which are detailed nominal descriptors 
            for billing and record-keeping purposes,
\end{ohlist}

\section*{Recent Comparison, St. Louis Children's Hospital}
\begin{ohlist}
    \item   Often these ordinal measures track closely, but have observable differences, relative to the GOS (Observed Glasgow Outcome Scale: 
            D, death; V, vegetative; SD, severe disabled; MD, moderate disabled; G, good recovery).\\
            \begin{Large}
            \begin{center}
            \begin{tabular}{lrrrrr}
                                        & \multicolumn{5}{c}{\emph{GOS Code}} \\
            \emph{Discharge Disposition}   & 1  & 2 &  3 &  4 &  5 \\
                                        \hline
            Medical Examiner/Morgue        & 9  & 0 &  0 &  0 &  0 \\
            Different Acute Care Hospital  & 0  & 0 &  4 &  0 &  0 \\
            Inpatient Rehab Facility       & 0  & 1 & 19 & 16 &  4 \\
            Home With Healthcare           & 0  & 0 &  1 &  0 &  0 \\
            Home With Outpatient Rehab     & 0  & 0 &  6 &  9 &  2 \\
            Home With No Assistance        & 0  & 0 &  9 & 32 & 11 \\
                                        \hline
            \end{tabular}
            \end{center}
            \end{Large}
    \item   Correlation:
            \begin{VM}
library(polycor)
polychor(head.inj3$Destination, head.inj3$GOS_hospital_disharge) # NOTICE THE "h"
[1] 0.6705342
            \end{VM}
\end{ohlist}

\section*{Ordered Outcomes Modeling}
\begin{ohlist}
    \item   The applied models are typically not \textcolor{MyEmph}{ordered probit} and \textcolor{MyEmph}{ordered logit}
            specifications, \emph{despite the obvious ordered nature of the data}.
    \item   Such models are standard tools for statisticians since Aitchison \& Silvey (1957) and Bock and Jones (1968, Ch.8), and
            introduced to social science statistics by McKelvey and Zavoina (1975). 
    \item   However, you can find \emph{no citations} to these models in clinical outcomes analysis until 1993.
    \item   Now of course they are more common, but not fully appreciated by all medical outcomes researchers.
\end{ohlist}

\section*{Threshold Approach}
\begin{ohlist}
        \item   \bgm$\exists\; \X$,\egm\ an \bgm$n \times k$\egm\ matrix of explanatory variables.
        \item   \bgm$Y$\egm\ observed on ordered/recorded on ordered categories: \bgm$Y_i \in [1,\ldots,k]$\egm, for \bgm$i=1,\ldots
                n$\egm.
        \item   \bgm$Y$\egm\ assumed to be produced by an unobserved (latent) variable \bgm$U$\egm.
        \item   \bgm$U$\egm\ is continuous on \bgm$\mathfrak{R}$\egm\ from \bgm$-\infty$ to $\infty$\egm.
        \item   The ``response mechanism'' for the \bgm$r^{th}$\egm\ category:
                \bgm\begin{equation}
                        Y=r \Longleftrightarrow \theta_{r-1} < U < \theta_r     \nonumber
                \end{equation}\egm
		        \vspace{-33pt}
        \item   This requires there to be thresholds on \bgm$\mathfrak{R}$\egm\ (no intercept):
        	    \bgm\begin{equation*}
        	    \U_i: \;
            		\theta_0 \underset{c=1}{\longleftarrow\!\longrightarrow}
            		\theta_1 \underset{c=2}{\longleftarrow\!\longrightarrow}
            		\theta_2 \underset{c=3}{\longleftarrow\!\longrightarrow}
            		\theta_3\ldots
            		\theta_{C-1} \underset{c=C}{\longleftarrow\!\longrightarrow}
            	    \theta_C        \nonumber
        	    \end{equation*}\egm
		        \vspace{-33pt}
        \item   The vector of (unseen) utilities across individuals in the sample, \bgm$\U$\egm, is determined by a linear
                additive specification of explanatory variables: \bgm$\U = -\X\B +\E$\egm, where \bgm$\B =
                [\beta_1,\beta_2,\ldots,\beta_p]$\egm\ does not depend on the \bgm$\theta_j$\egm, and \bgm$\E \sim F_{\E}$\egm.
\end{ohlist}

\newpage
\section*{Threshold Approach}
\begin{ohlist}
    \item   For the observed vector $\Y$:
            \bgm\begin{align*}
                        p(\Y \le r|\X) &= p(\U \le \theta_r) = p(-\X\B + \E \le \theta_r)       \9
                                       &= p(\E \le \theta_r+\X\B) = F_{\E}(\theta_r + \X\B).
            \end{align*}\egm
		    \vspace{-33pt}
	\item	This is called the \emph{cumulative model} because:
		    \bgm\begin{equation*}
				p(\Y \le \theta_r|\X) = p(\Y=1|\X) + p(\Y=2|\X) + \ldots + p(\Y=r|\X)
		    \end{equation*}\egm
		    \vspace{-33pt}
   	\item   A logistic distributional assumption on the errors produces the ordered logit specification:
        	\bgm\begin{equation*}
            		F_{\E}(\theta_r - \X'\B) = P(\Y\le r|\X) = [1+\exp(-\theta_r-\X'\B)]^{-1}
        	\end{equation*}\egm
		    \vspace{-33pt}
    \item   The likelihood function is:
            \bgm\begin{equation*}
                        L(\B,\T|\X,\Y) = \prod_{i=1}^{n}\prod_{j=1}^{C-1}\left[
                    		\Lambda(\theta_j + \X_i'\B) - \Lambda(\theta_{j-1} + \X_i'\B) \right]^{z_{ij}}
            \end{equation*}\egm
		    \vspace{-33pt}
		    where \bgm$z_{ij}=1$\egm\ if the \bgm$i$\egm th case is in the \bgm$j$\egm th category, and \bgm$z_{ij}=0$\egm\ otherwise.
\end{ohlist}

\section*{Interpretation}
\begin{ohlist}
	\item	Predicted probabilities use: \bgm$ P(\Y\le r|\X) = [1+\exp(-\theta_r-\X'\B)]^{-1} $\egm\ after estimation of
            \bgm$\B$\egm.
\end{ohlist}
\begin{center}
   \epsfig{file=/Users/jgill/ARTICLES/Article.PTBI/ordered.logit.fig.ps,height=7.6in,width=4.6in, clip=,angle=270}
\end{center}
\begin{comment}
postscript("Article.PTBI/ordered.logit.fig.ps")
theta <- c(-2,-0.5,3,4,7)
ruler <- seq(-8,8,length=300)
par(col.axis="white",col.lab="white",col.sub="white", col="white", bg="slategray",cex.lab=1.5,mar=c(6,6,2,2))
plot(ruler,exp(ruler)/(1+exp(ruler)),type="l",lwd=3, col="yellow",xlab="Predictor",ylab="Probability",ylim=c(-0.1,1))
for (i in 1:length(theta))  {
    segments(theta[i],0,theta[i],exp(theta[i])/(1+exp(theta[i])), col="black",lwd=1.5)
    text(expression(theta),x=theta[i],y=-0.07,col="black",cex=1.5)
    text(i,x=(theta[i]+0.2),y=-0.08,col="black",cex=1.0)
}
abline(h=0,col="white")
lines(ruler,exp(ruler)/(1+exp(ruler)),lwd=3, col="yellow")
dev.off()
\end{comment}

\section*{Proportional-Odds Calculation for Ordered Logit}
\noindent Rewrite According to:
\bgm\begin{align*}
	p(\Y \le \theta_r|\X) 	&= \frac{ \exp(\theta_r+\X\B) }{ 1 + \exp(\theta_r+\X\B) }	\9
	\intertext{\textcolor{black}{and:}}
	p(\Y > \theta_r|\X)	&= \frac{ 1+ \exp(\theta_r+\X\B) }{ 1 + \exp(\theta_r+\X\B) } 
	                         - \frac{ \exp(\theta_r+\X\B) }{ 1 + \exp(\theta_r+\X\B) }	\9
	                        &= \frac{ 1 }{ 1 + \exp(\theta_r+\X\B) }	
\end{align*}\egm
so:
\bgm\begin{equation*}
	\frac{ p(\Y \le \theta_r|\X) }{ p(\Y > \theta_r|\X) } 
		= \frac{ \exp(\theta_r+\X\B) / (1 + \exp(\theta_r+\X\B)) }
			{ 1 / (1 + \exp(\theta_r+\X\B)) }
		= \exp(\theta_r+\X\B)
\end{equation*}\egm
which is nice.  And:
\bgm\begin{equation*}
	\log\left[ \frac{ p(\Y \le \theta_r|\X) }{ p(\Y > \theta_r|\X) } \right] = \theta_r+\X\B
\end{equation*}\egm
which is nicer.

\section*{Pediatric Neurocritical Care Program (Pineda \emph{etal}, 2013)}
\begin{ohlist}
    \item   10 years of PTBI STL Children's Hospital data with a change in the middle-point (September 2005).
    \item   PNCP: a time-sensitive, severity-based approach to monitor and treat children with TBI that coordinated communication 
            and activity amongst PICU staff and physician faculty and trainees, conforming with the 2003 Brain Trauma Foundation
            guidelines.
    \item   This included a detailed training program, an explicit process for maintaining pathway fidelity, and continuous 
            quality improvement. 
    \item   Groups: \bgm$n_{Pre-PNCP}=63$\egm, \bgm$n_{Post-PNCP}=60$\egm, treated as a fixed effect variable (treatment contrast).
    \item   Tests for differences in demographics between the two periods failed to find statistically reliable differences.
    \item   Outcomes: Medical Examiner/Morgue, Different Acute Care Hospital, Inpatient Rehab Facility, Home With Healthcare,
            Home With Outpatient Rehab, Home With No Assistance (better in the positive direction).
\end{ohlist}

\section*{Results from the Ordered Probit Model}
\begin{large}
\begin{center}
\begin{tabular}{lrrr}
                               &  Coefficient   & Std.Err.  & t-value \\
                               \hline
    Post-PNCP                  &     0.482477   & 0.216061  &   2.233 \\
    Age In Months              &    -0.004674   & 0.002127  &  -2.198 \\
    White                      &    -0.318926   & 0.129315  &  -2.466 \\
    Length of Stay in PICU     &    -0.003776   & 0.007839  &  -0.482 \\
    Male                       &     0.111984   & 0.107548  &   1.041 \\
    ICP Monitoring             &     0.997479   & 0.299579  &   3.330 \\
    Post-Resuscitation GCS     &     0.125677   & 0.060159  &   2.089 \\
    PRISM III                  &    -0.065137   & 0.018125  &  -3.594 \\
    Injury Severity Score\^{}2 &    -0.000315   & 0.000134  &  -2.345 \\
    Fall                       &     0.291087   & 0.268258  &   1.085 \\
    Motor Vehicle Accident     &     0.197797   & 0.191271  &   1.034 \\
    Pedestrian Accident        &     0.147976   & 0.241442  &   0.613 \\
                               \hline
\end{tabular}
\end{center}
\end{large}

\vspace{0.7in}
\noindent NOTES:
\begin{ohlist}
    \item   Reference category for the injury etiologies is ``Other.''
    \item   Race (white) \texttt{-0.318926}, means that moving from 0=non-white to 1=white pushed the expected outcome down 
            the scale of \bgm$U$\egm\ towards more unfavorable outcomes.
    \item   Coefficients such as ICP Monitoring \texttt{0.997479}, have the opposite effect. 
\end{ohlist}

\section*{Ordered Probit Threshold Estimates}
\begin{large}
\begin{center}
\begin{tabular}{llrrr}
    Threshold  & Categories Separated                                             & Coefficient & Std. Error    & t-value \\
    \hline
    \bgm$\theta_1$\egm\ & Medical Examiner/Morgue \emph{to} Different Acute Care Hospital  & -0.647      & 0.188         & -3.442 \\
    \bgm$\theta_2$\egm\ & Different Acute Care Hospital \emph{to} Inpatient Rehab Facility & -0.377      & 0.226         & -1.670 \\
    \bgm$\theta_3$\egm\ & Inpatient Rehab Facility \emph{to} Home With Healthcare          &  0.979      & 0.262         &  3.733 \\
    \bgm$\theta_4$\egm\ & Home With Healthcare \emph{to} Home With Outpatient Rehab        &  1.005      & 0.262         &  3.829 \\
    \bgm$\theta_5$\egm\ & Home With Outpatient Rehab \emph{to} Home With No Assistance     &  1.433      & 0.266         &  5.391 \\
    \hline
\end{tabular}
\end{center}
\end{large}

\vspace{1in}
\noindent NOTES:
\begin{ohlist}
    \item   The literal value of these coefficients is unimportant.
    \item   The statistical significance of these coefficients is unimportant.
    \item   They are important only to ``help'' estimate the \bgm$\boldsymbol{\beta}$\egm\ coefficients.  
\end{ohlist}

\section*{Smooth Predictions From the Model}
\begin{center}
   \epsfig{file=../Article.PTBI/outcomes.graph.col.ps,height=8.6in,width=5.6in, clip=,angle=270}
\end{center}

\section*{Making a Prediction Difference Using Race}
\begin{ohlist}
    \item   Suppose a predicted outcome on the \bgm$U$\egm\ metric for a particular non-white patient was \bgm$-0.4$\egm\ 
            (given values for all of the other \bgm$\mathbf{X}$\egm\ variables).  
    \item   Then the model would predict the category [Different Acute Care Hospital].  
    \item   If this was changed to a white patient, the coefficient \bgm$\beta_{\text{Race}}$\egm\ would assign \bgm$-0.318926$\egm\ 
            instead of \bgm$0$\egm\ in the \bgm$\mathbf{X}_i\boldsymbol{\beta}$\egm\ calculation.  
    \item   This reduction gives the (hypothetical) patient a prediction of \bgm$U_i = -0.718926$\egm, which corresponds to 
            the categorical prediction of [Medical Examiner/Morgue].
\end{ohlist}

\section*{Graphical Comparison}
\begin{tabular}{lcr}
\parbox{0.25\linewidth}{
    \begin{ohlist}
        \item   The x-axis is the \bgm$U$\egm\ metric and the y-axis is probability.  
        \item   The five \bgm$\theta$\egm\ cutpoints are given by the dotted vertical lines and labeled at the top. 
        \item   The slices give the probability for being in each of the categories for a white male following 
                a motor vehicle accident, with all other explanatory variables set at the data mean.
    \end{ohlist}
}
& $\quad$ &
\hspace{-0.5in}
\parbox{0.69\linewidth}{
    \epsfig{file=../Article.PTBI/ordered.probit.ps,width=8.5in,height=5.5in,clip=,angle=0}
}
\end{tabular}

\begin{comment}
\section*{Analysis of First Differences}
\begin{ohlist}
    \item   Select two levels of one explanatory variable and setting all others at their means: 
            \bgm$\mathbf{X}_1$ and $\mathbf{X}_2$\egm.  
    \item   Two hypothetical probability vectors are created by applying the link function to
            \bgm$\mathbf{X}_1\boldsymbol{\beta}$\egm\
            and \bgm$\mathbf{X}_2\boldsymbol{\beta}$\egm, which can be compared.  
    \item   Race is set at white, etiology is set at motor vehicle accident, and the two vectors are then made 
            different with the Group variable: one indicates Pre-PNCP and the other indicates Post-PNCP status.  
    \item   All other variables except Group, Race and Etiology are set at the data means.
    \item   These almost identical cases are multiplied by the estimated regression coefficient vector 
            and the ordered probit link function then transforms each onto the probability scale for comparison.  
    \item   The probability of death falls 53\% following PNCP initiation (from \bgm$0.210$\egm\ to \bgm$0·0988$\egm) 
            and the probability for discharge to home with no assistance increases 53\% (from \bgm$0.101$\egm\ to 
            \bgm$0.214$\egm). 
\end{ohlist}
\end{comment}

\section*{Clinical Status of PTBI Patients}
\begin{ohlist}
    \item   \emph{Secondary insults} continue to be reported in pediatric patients with severe TBI where intracranial hypertension
            (ICH) is the most common one. 
    \item   These can result in direct brain injury and even cerebral herniation and death, or contribute to low cerebral perfusion, 
            worsening brain metabolism. 
    \item   Current recommendations for initiation of intracranial pressure (ICP) monitoring and management of ICH are based on 
            clinical and radiological indicators of injury severity. 
            %(Glasgow Coma Scale score and computerized tomography) and ICH thresholds associated with worse outcome in a time-dose dependent fashion [REF]. 
    \item   Yet pediatric randomized trials lowering ICH have not demonstrated a benefit on outcomes. 
\end{ohlist}

\section*{Current Biomarker Theoretical Basis}
\begin{ohlist}
    \item   We propose a focus on neuropathological information from two serum based biomarkers of cellular injury: 
            \begin{ohlist2}
                \item   \textcolor{MyEmph}{ubiquitin carboxyl-terminal esterase L1} (UCH-L1) 
                \item   \textcolor{MyEmph}{glial fibrillary acidic protein} breakdown products (GFAP-BDPs) 
            \end{ohlist2}
            that can improve our ability to characterize injury progression in pediatric severe TBI patients. 
    \item   These biomarkers may in the future also allow biological quantification of response to therapy, facilitating timely and 
            individualized adjustments in patient care and consequently better outcomes.
\end{ohlist}

\section*{UCH-L1}
\begin{ohlist}
    \item   A highly abundant neuronal protein thought to play a critical role in cellular protein degradation during both normal and 
            pathological conditions. 
    %\item   This enzyme is a thiol protease that recognizes and hydrolyzes a peptide bond at the C-terminal glycine of ubiquitin.              
    \item   This soluble protein constitutes up to 10\% of cytoplasmic protein in neurons, is elevated in cerebrospinal fluid (CSF) 
            and serum following TBI, and is significantly associated with measures of injury severity and outcome in adults and 
            children. 
    \item   Localized to neurons in the cerebral cortex, and has demonstrated anecdotal accuracy in stratifying adult patients 
            with varying degrees of TBI within an hour of injury in addition to predicting outcomes.
\end{ohlist}

\section*{GFAP}
\begin{ohlist}
    \item   A type III intermediate filament that forms part of the cytoskeleton of mature astrocytes and other glial cells. 
    \item   GFAP is only found in the central nervous system, and it presence in the blood is a marker of astrogial injury.  
    \item   Central nervous system (CNS) injury (only) causes gliosis and subsequently up-regulates blood GFAP, making it a
            potential  marker of TBI as opposed to other trauma.
    \item   Specifically, GFAP is a product of astrocyte cytoskeleton degradation by calpain protease (calcium-dependent,
            non-lysosomal) activation and therefore considered specific to the CNS. 
    \item   Elevated concentrations of GFAP two days post injury portend a poor prognosis and are thought to reflect the presence 
            of a secondary insults (i.e. ICH and low brain tissue oxygen tension)
\end{ohlist}

\section*{The Data}
\begin{ohlist}
    \item   A prospective case-control cohort study, \bgm$n=67$\egm\ pediatric patients (17 years old and younger) admitted to 
            the Intensive Care Unit at Saint Louis Children’s Hospital with severe TBI (post-resuscitation GCS) score 8 or less (12). 
    \item   In the case of suspected non-accidental trauma, children with computerized tomography evidence of old intracranial hemorrhage 
            were excluded. 
    \item   Dr. Rachel Berger and colleagues at the University of Pittsburgh provided control samples from a cohort of 30 normal children 
            either as part of a 1-year wellness check-up or during the workup of an acute injury for which TBI was ruled out.
    \item   Blood plasma was collected from these control patients admitted to the Pediatric Intensive Care Unit with normal mental status and 
            no history of TBI (GCS = 15).  
    \item   Blood was sampled at regular time points following injury (6, 24, 48, 72 hours) and centrifuged for 10 minutes at 5,000 
            revolutions per minute and serum from these samples was stored at -80C until the time of assay.  
\end{ohlist}

\section*{ROC Curves}
\vspace{-2.0in}
\begin{center}
\parbox{\linewidth}{  \begin{center} 
    \epsfig{file=./roc.ps, height=10.5in,width=7.25in,clip=,angle=270}
    \end{center} }
\end{center}

\section*{Discharge Model, Biomarkers at 24 Hours}
\begin{VM}
biom1 <- polr(as.factor(GOSE.hospital.D.C) ~ (UCHL1.ng.ml.24hrSerum) 
                                           + (GFAP.ng.ml.24.hr.Serum)
    + Age.mo + Post.Res.GCS + NAT
    , data=biomarkers.stl[1:67,])

Coefficients:
                            Value Std. Error t value
UCHL1.ng.ml.24hrSerum  -0.3802111  0.2991090 -1.2711
GFAP.ng.ml.24.hr.Serum -0.0542211  0.0249381 -2.1742
Age.mo                 -0.0091644  0.0064066 -1.4305
Post.Res.GCS            0.3433188  0.1921960  1.7863
NAT                    -3.0600989  1.1653334 -2.6259
\end{VM}

\section*{Discharge Model, Biomarkers at 48 Hours}
\begin{VM}
biom2 <- polr(as.factor(GOSE.hospital.D.C) ~ (UCHL1.ng.ml.48hrSerum) 
                                           + (GFAP.ng.ml.48hrSerum)
    + Age.mo + Post.Res.GCS + NAT
    , data=biomarkers.stl[1:67,])

Coefficients:
                          Value Std. Error  t value
UCHL1.ng.ml.48hrSerum -0.489430  0.2426552 -2.01698
GFAP.ng.ml.48hrSerum  -0.097880  0.0272559 -3.59115
Age.mo                -0.007701  0.0059658 -1.29085
Post.Res.GCS           0.070040  0.1796915  0.38978
NAT                   -0.296157  1.3070076 -0.22659
\end{VM}

\section*{Discharge Model, Biomarkers at 72 Hours}
\begin{VM}
biom3 <- polr(as.factor(GOSE.hospital.D.C) ~ (UCHL1.ng.ml.72hrSerum) 
                                           + (GFAP.ng.ml.72hrSerum)
    + Age.mo + Post.Res.GCS + NAT
    , data=biomarkers.stl[1:67,])

Coefficients:
                          Value Std. Error  t value
UCHL1.ng.ml.72hrSerum -1.448308  0.6202287 -2.33512
GFAP.ng.ml.72hrSerum  -0.188987  0.0586528 -3.22212
Age.mo                -0.018231  0.0091357 -1.99555
Post.Res.GCS          -0.203829  0.3367397 -0.60530
NAT                   -0.384292  1.6886275 -0.22758
\end{VM}




} % END OF \color{white}{
\end{Large}\end{landscape}\end{document}
